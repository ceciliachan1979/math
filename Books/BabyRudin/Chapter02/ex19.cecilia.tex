\subsection*{Exercise 19 (Cecilia)}
\subsubsection*{Part a}
Since $ A $ and $ B $ are closed, $ \overline{A} = A $ and $ \overline{B} = B $, so $ A \cap B = \phi $ implies $ \overline{A} \cap B = \phi $ and $ A \cap \overline{B} = \phi $, in other words, $ A $ and $ B $ are separated.
\subsection*{Part b}
Consider a point in $ \overline{A} - A $, it is a limit point of $ A $, so every neighborhood of it contains a point in $ A $, so it is impossible to be an interior point of $ B $, therefore $ \overline{A} \cap B = \phi $. Similarly, $ A \cap \overline{B} = \phi $, so $ A $ and $ B $ are separated.
\subsection*{Part c}
$ A $ is a neighborhood, so it is open. Any point in $ B $ is an interior point of $ B $ because if $ q \in B $, then $ d(p, q) > \delta $, so the neighborhood of $ q $ with radius $ \delta - d(p, q) $ is contained in $ B $. Therefore, $ B $ is open. Since $ A $ and $ B $ are both open and they are disjoint, they are separated by part b.
\subsection*{Part d}
For any real number $ 0 < \delta < d(p, q) $, part c shows that there are two separated sets $ A $ and $ B $. These sets are non-empty because $ p \in A $ and $ q \in B $. The space is connected, so there must be another point $ r $ in the space such that $ r \notin A $ and $ r \notin B $. Therefore $ d(p, r) = \delta $. We can pick this point for each $ \delta $.
This shows that the space has an uncountable subset and therefore the space is uncountable.