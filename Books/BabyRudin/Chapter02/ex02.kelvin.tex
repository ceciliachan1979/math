\subsection*{Exercise 2 (Kelvin)}
Consider the following set:

$\forall{N} \in \mathbb{Z^+}, A_N = \{a_0z^n + a_1z^{n-1} + ... + a_n, z \in \mathbb{C}\mid n + |a_0| + |a_1| + ... + |a_n| = N\}$.

Given that $n + |a_0| + |a_1| + ... + |a_n| = N$ is finite, which means we can only have finite number of $a_i$, for each $\text{$i$ from 0 to $n$} $, 
therefore $A_N$ is also finite, 
and $\bigcup_{N \in \mathbb{Z+}}^{\infty} A_N$ is at most countable as countable union of finite set is at most countable, which means the set of all complex polynomials with integer coefficient is at most countable.

Let consider the roots of a particular polynomial in $A_N$, denoted as $B_N$:

$\forall{N} \in \mathbb{Z^+}, B_N = \{z \in \mathbb{C}\mid f(z) = 0, \exists{f} \in A_N \}$

Since the number of roots of a polynomial is finite, therefore $B_N$ is finite, and $\bigcup_{N \in \mathbb{N+}}^{\infty} B_N$ is at most countable.

To show a set C = $\bigcup_{N \in \mathbb{N+}}^{\infty} B_N$ is countable, we need to show set C is infinite. Consider for every integer a $\in \mathbb{Z}$,
we can construct a polynomial $z - a = 0 \implies z = a$, which means the $\mathbb{Z}$ is a subset of C. Since $\mathbb{Z}$ is infinite, C is at most countable and
have a infinite subset, C is countably infinite.
