\subsection*{Exercise 15 (Cecilia)}
To show that the theorem is not true for closed subsets, consider the family of set $ \mathbb{N} - \{n\} $ for $ n \in \mathbb{N} $ as subsets of the real number line.

All the sets are closed because these isolated points have no additional limit points. (i.e any points that is not a natural number will have a neighborhood that does not intersect with the set.)
Any finite intersection is not empty because there is a maximum number that is being deleted by the sets, and the number after it is not deleted by any set and therefore belong to the intersection.
But the infinite intersection is empty because there is no maximum number that is not deleted by any set.

To show that the theorem is not true for bounded subsets, consider the family of set $ [1, 2] $ with $ r $ deleted for all real numbers $ r $.

All the sets are bounded because they are all subsets of $ [1, 2] $.
Any finite intersection is not empty because there are infinitely many real numbers in $ [1, 2] $ but we only deleted a finite number of them, so there must be a number that is not deleted.
But the infinite intersection is empty because we deleted all the real numbers in $ [1, 2] $.

