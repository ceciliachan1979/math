\subsection*{Exercise 05 (Gapry)}

\begin{flushleft}

Define a set $E = \{\frac{1}{n} \ | \ n \in Z^{+}\}$, we need to find the
limit points of it.
\vspace{10px}

\textbf{Case 1: n is a finite number} \\
$\exists\ r > 0$, $(\frac{1}{n} - r, \frac{1}{n} + r) \cap \{\frac{1}{n}\}
                                                         = \{\frac{1}{n}\}$,
hence, $\frac{1}{n}$ is not the limit point of $E$ if $n$ is finite.
\vspace{10px}

\textbf{Case 2: n is infinite} \\
If $n \to \infty$, $\frac{1}{n} = 0$, we need to show 
$\forall\ r > 0$, $(\frac{1}{n} - r, \frac{1}{n} + r) \cap \{\frac{1}{n}\} 
\neq \emptyset$ or $(\frac{1}{n} - r, \frac{1}{n} + r) \cap \{\frac{1}{n}\} 
\neq \{\frac{1}{n}\}$, that is $\exists\ n \in Z^{+}$ s.t. $\frac{1}{n} \in 
(-r, r)$. 
\vspace{10px}

According to the Archimedean property, we know $nr > 1, n \in Z^{+}$, hence
$\frac{1}{n} < r$. Since $n > 0$, we know $-r < \frac{1}{n} < r$ is true. 
Therefore, $0$ is the limit point of $E = \{\frac{1}{n} \ | \ n \in Z^{+}\}$.
\vspace{10px}

Define a set $E_k = \{\frac{1}{n} + k \ | \ n \in Z^{+}, k \in Z^{+}\}$. Since
we know $-r < \frac{1}{n} < r$ is true, $-r + k < \frac{1}{n} + k < r + k$ is 
also true. That is, $\forall\ r > 0$, $\exists\ n \in Z^{+}$ s.t. 
$\frac{1}{n} + k \in (-r + k, r + k)$ if n is infinite, k is finite. 
Therefore, $k$ is the limit point of $E_k$.
\vspace{10px}

Let's define three sets $E_{c_1}$, $E_{c_2}$ and $E_{c_3}$ \\
$E_{c_1} = \{\frac{1}{n} + c_1 \ | \ n \in Z^{+}, c_1 \in Z^{+}\}$ \\
$E_{c_2} = \{\frac{1}{n} + c_2 \ | \ n \in Z^{+}, c_2 \in Z^{+}\}$ \\
$E_{c_3} = \{\frac{1}{n} + c_3 \ | \ n \in Z^{+}, c_3 \in Z^{+}\}$ \\
\vspace{10px}

We know the limit point of corresponding set are $c_1$, $c_2$ and $c_3$, 
if $E = E_{c_1} \cup E_{c_2} \cup E_{c_3}$ and $c_1 \neq c_2 \neq c_3$,
it's a bounded set of real numbers with exactly three limit points which are
$c_1 \neq c_2 \neq c_3$. Q.E.D.

\end{flushleft}