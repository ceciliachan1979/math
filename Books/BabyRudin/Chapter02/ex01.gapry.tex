\subsection*{Exercise 01 (Gapry)}

\begin{flushleft}
\textbf{Case 0:} The empty set is a subset of the empty set.

\textit{Proof by definition}, $A \subseteq B$ if and only if $A \cap B = A$.

Since $\emptyset \cap \emptyset = \emptyset$, it follows that $\emptyset \subseteq \emptyset$. 
\end{flushleft}

\begin{flushleft}
\textbf{Case 1:} The empty set is a subset of any non-empty set.

\textit{By definition}, if set $A$ is a subset of $B$, then $\forall x \in A$ such that $x \in B$. Conversely, if set $A$ is not a subset of $B$, then $\exists x \in A$ such that $x \notin B$.

\textit{Proof by contradiction}, assume the empty set is not a subset of a non-empty set $B$. Then, $\exists x \in \emptyset$ such that $x \notin B$. However, the empty set does not contain any elements, which contradicts our assumption.
\end{flushleft}

\begin{flushleft}
By applying \textbf{Case 0} and \textbf{Case 1}, we can conclude that the empty set is a subset of every set.
\end{flushleft}
