\subsection*{Exercise 01}
The set of rational numbers $\mathbb{Q}$ has the following properties:
\begin{itemize}
    \item \textbf{Closure Property:} For any two elements $a, b \in \mathbb{Q}$, their sum $a + b$ and product $a \cdot b$ are also in $\mathbb{Q}$.
    \item \textbf{Inverse Property:} For any element $a \in \mathbb{Q}$, its additive inverse $-a$ and multiplicative inverse $a^{-1}$ are also in $\mathbb{Q}$.
\end{itemize}

\begin{flushleft}
\textbf{Part a)} Proof by Contradiction: Assume $r + x \in \mathbb{Q}\ $ for $r \in \mathbb{Q}$ and $x \in \mathbb{R} \setminus \mathbb{Q}$.
\begin{enumerate}
    \item $r \in \mathbb{Q}$
    \item $-r \in \mathbb{Q}$ (By Inverse Property).
    \item $-r + r + x \in \mathbb{Q}$ (By Closure Property), which implies $x \in \mathbb{Q}$. 
    \\
    This contradicts the assumption that $x \in \mathbb{R} \setminus \mathbb{Q}$.
\end{enumerate}
\end{flushleft}

\begin{flushleft}
\textbf{Part b)} Proof by Contradiction: Assume $rx \in \mathbb{Q}\ $ for $r \in \mathbb{Q}$ and $x \in \mathbb{R} \setminus \mathbb{Q}$.
\begin{enumerate}
    \item $r \in \mathbb{Q}$.
    \item $r^{-1} \in \mathbb{Q}$ (By Inverse Property).
    \item $(r^{-1} \cdot rx) \in \mathbb{Q}$ (By Closure Property), which implies $x \in \mathbb{Q}$. 
    \\
    This contradicts the assumption that $x \in \mathbb{R} \setminus \mathbb{Q}$.
\end{enumerate}
\end{flushleft}

\begin{flushleft}
Since $r + x$ and $rx$ are not the rational numbers, \\
they must be irrational numbers. 
\end{flushleft}