\subsection*{Exercise 05 (Kelvin)}
Due to the fact that the $\mathbb{R}$ has the greatest lower bound property, $ \inf(A) $ exists.\\

$\forall x \in A, x \ge \inf(A) \implies -x<-\inf(A) \implies -\inf(A)$ is an upper bound of $-A$.\\

To verify whether $-\inf(A)$ is the least upper bound of $-A$ ($-\inf(A) = \sup(-A)$), we check that any real number less than it is not an upper bound.

$\forall \gamma \in \mathbb{R} $ such that $ \gamma < -\inf(A) $, we know $ -\gamma > \inf(A)$ \\

Given that $\inf(A)$ is the greatest lower bound, therefore, $ \exists x \in A$ such that $ -\gamma > x $.

With that, we have $ \gamma < -x $. so indeed $\gamma$ is not an upper bound of $-A$. \\

Therefore, $-\inf(A) = \sup(-A) \implies \inf(A) = -\sup(-A)$.
