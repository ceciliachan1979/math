\subsection*{Exercise 05 (Kelvin)}
Due to the fact that the $\mathbb{R}$ has the greatest lower bound property, the Infimum of $A$ exists: $ \alpha = \inf A $.\\
This implies:

$\forall x \in A, x \ge \alpha \implies -x<-\alpha \implies -\inf A$ is the upper bound of $-A$.\\
To verify whether $-\inf A$ is an least upper bound of $-A$ ($-\inf A = \sup -A$),

$\forall \gamma \in \mathbb{R}, \gamma < -\inf A \implies -\gamma > \inf A$ \\
Given that $\inf A$ is the greatest lower bound, from $-\gamma > \inf A$, we have

$\exists x \in A, \forall \gamma \in \mathbb{R}, -\gamma > x \implies \gamma < -x $

$\implies \exists x \in -A, \forall \gamma \in \mathbb{R}$, $\gamma$ is not an upper bound of $-A$. \\
Therefore, $-\inf A = \sup -A \implies \inf A = -\sup A$.
