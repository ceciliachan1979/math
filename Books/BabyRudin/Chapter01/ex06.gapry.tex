\subsection*{Exercise 06 (Gapry)}

\subsubsection*{6.(a)}

We know the equation $r = \frac{m}{n} = \frac{p}{q}$ by assumption,
it follows $nqr = mq = pn$ (Closure Axiom, Existence Multiplication Inverse Element). According to Theorem 1.21, there is one and only one positive real $y$ such that
$y^{nqr} = y^{mq} = y^{pn} = x$ if $nqr = mq = pn$. In the case, $y$ equals $b$ so the equation turns to $b^{nqr} = b^{mq} = b^{pn}$, $b > 1$. Let's analyze $((b^m)^\frac{1}{n})^{nq}$ and $((b^p)^\frac{1}{q})^{nq}$

\begin{flalign*}
((b^m)^\frac{1}{n})^{nq} &= (((b^m)^\frac{1}{n})^n)^q           \text{ (Exponent rules, $(b^i)^j = b^{ij}$, i, j are integers)} &\\
                         &= (b^m)^q                             \text{ (by definition)} &\\
                         &= b^{mq}                              \text{ (Exponent rules,
                                                                        $(b^i)^j = b^{ij}$,
                                                                        i, j are integers)} &\\
((b^p)^\frac{1}{q})^{nq} &= ((b^p)^\frac{1}{q})^{qn}            \text{ (by 1.12 (M2))} &\\
                         &= (((b^p)^\frac{1}{q})^q)^n           \text{ (Exponent rules,
                                                                        $(b^i)^j = b^{ij}$,
                                                                        i, j are integers)} &\\
                         &= (b^p)^n                             \text{ (by definition)} &\\
                         &= b^{pn}                              \text{ (Exponent rules,
                                                                        $(b^i)^j = b^{ij}$,
                                                                        i, j are integers)}
\end{flalign*}

According to Theorem 1.21, there is one and only one positive real $y$ such that $y^{nq} = b^{mq} = b^{pn} = x$, hence
$y = (b^m)^\frac{1}{n} = (b^p)^\frac{1}{q}$.

\subsubsection*{6.(b)}
Let's claim $r = \frac{p_1}{q_1}$, $s = \frac{p_2}{q_2}$, $p_1$, $q_1$, $p_2$, $q_2$ are integers.
\begin{flalign*}
    r + s &= \frac{p_1}{q_1} + \frac{p_2}{q_2}
           = \frac{p_1q_2 + p_2q_1}{q_1q_2}
           = (p_1q_2 + p_2q_1) \cdot \frac{1}{q_1q_2}       \text{} &\\
b^{r + s} &= b^{(p_1q_2 + p_2q_1) \cdot \frac{1}{q_1q_2}}   \text{} &\\
          &= (b^{p_1q_2} \cdot b^{p_2q_1})^\frac{1}{q_1q_2} \text{ (Exponent rules,
                                                                    $b^{i + j} = b^i \cdot b^j$,
                                                                    i, j are integers)} &\\
          &= b^{p_1q_2 \cdot \frac{1}{q_1q_2}} \cdot
             b^{p_2q_1 \cdot \frac{1}{q_1q_2}}              \text{ (1.21 Corollary)} &\\
          &= b^{\frac{p_1}{q_1}} \cdot b^{\frac{p_2}{q_2}}  \text{} &\\
          &= b^{r} \cdot b^{s}
\end{flalign*}
