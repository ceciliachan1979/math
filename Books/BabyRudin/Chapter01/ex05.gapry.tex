\subsection*{Exercise 05 (Gapry)}
\begin{flushleft}
Since $A \subset R$, and $A$ is bounded below, we know that $\inf A$ exists.\\
\vspace{10px}
Let's claim $\alpha = \inf A$, so for all $x \in A$, we have $x \ge \alpha$. It follows that for all $-x \in -A$, we have $-x \le -\alpha$.\\
\vspace{10px}
Since we know that $\alpha$ is the greatest lower bound of $A$, the negation of inequality $x \ge \alpha$, which is $-x \le -\alpha$, it shows $-\alpha$ is the least upper bound of $-A$, that is, $-\alpha = \sup(-A)$. \\
\vspace{10px}
According to 1.14 Proposition (d), $\alpha = -(-\alpha)$, we can conclude that $\inf A = -\sup(-A)$.
\end{flushleft}
