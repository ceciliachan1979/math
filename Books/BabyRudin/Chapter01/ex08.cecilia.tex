\subsection*{Exercise 08 (Cecilia)}
After definition 1.26, Rudin identifies the complex number $ (a, 0) $ as the real number $ a $ on the basis that arithmetic of $ (a, 0) $ and $ a $ are compatible. However, that does not say anything about the ordering of the complex numbers. 
In fact, suppose an ordering exist, it might be the case that the ordering of $ (a, 0) $ is not compatible with the ordering of $ a $, so identification could be confusing. In the sequel, we will not use the identification, and use the ordered pair representation instead.

Suppose (for contradiction) that an ordering of complex number can be defined.

First, we show that $ (-1, 0) > (0, 0) $ is false. Suppose $ (-1, 0) > (0, 0) $, we have:

\begin{eqnarray*}
             (-1, 0) &>& (0, 0)          \\
    (-1, 0) + (1, 0) &>& (0, 0) + (1, 0) \\
              (0, 0) &>& (1, 0)
\end{eqnarray*}

\begin{eqnarray*}
           (-1, 0) &>& (0, 0)        \\
    (-1, 0)(-1, 0) &>& (0, 0)(-1, 0) \\
            (1, 0) &>& (0, 0)
\end{eqnarray*}

Apparently these two facts contradict each other. Therefore, $ (-1, 0) > (0, 0) $ is false.

With that result, suppose the complex number can be assigned an order so that it becomes an ordered field. Then we know that either $ i > (0, 0) $ or $ i < (0, 0) $.

Suppose $ i > (0, 0) $, then $ i^2 > (0, 0) $, which means $ (-1, 0) > (0, 0) $, which is false.

Otherwise suppose $ i < (0, 0) $, then $ (0, 0) = i - i < -i $, which means $ (0, 0) < (-i)^2 $, which means $ (0, 0) < (-1, 0) $, which is false.

Therefore we have a contradiction and it is impossible to impose an order on the complex numbers so that it becomes an ordered field.