\subsection*{Exercise 03 (Gapry)}

\subsubsection*{1.15 Proposition (a)}
As we know $x \neq 0$
\begin{flalign*}
                        xy &= xz                                       &\\
\implies \frac{1}{x}{x}{y} &= \frac{1}{x}{x}{z} \text{ (by 1.12 (M5))} &\\
\implies       1 \cdot {y} &= 1 \cdot {z}       \text{ (by 1.12 (M5))} &\\
\implies               {y} &= {z}               \text{ (by 1.12 (M4))}
\end{flalign*}

\subsubsection*{1.15 Proposition (b)}
As we know $x \neq 0$
\begin{flalign*}
                        xy &= x                                     &\\
\implies \frac{1}{x}{x}{y} &= \frac{1}{x}{x} \text{ (by 1.12 (M5))} &\\
\implies       1 \cdot {y} &= 1              \text{ (by 1.12 (M5))} &\\
\implies               {y} &= 1              \text{ (by 1.12 (M4))}
\end{flalign*}

\subsubsection*{1.15 Proposition (c)}
As we know $x \neq 0$
\begin{flalign*}
                        xy &= 1                                  &\\
\implies \frac{1}{x}{x}{y} &= \frac{1}{x} \text{ (by 1.12 (M5))} &\\
\implies       1 \cdot {y} &= \frac{1}{x} \text{ (by 1.12 (M5))} &\\
\implies               {y} &= \frac{1}{x} \text{ (by 1.12 (M4))}
\end{flalign*}

\subsubsection*{1.15 Proposition (d)}
We know that $x \in F$ and $x \neq 0$. According to 1.12, there must exist an inverse element $\frac{1}{x} \in F$ such that $x \cdot \frac{1}{x} = 1$. Similarly, since $\frac{1}{x} \in F$, there must exist an inverse element $\frac{1}{\frac{1}{x}} \in F$ such that $\frac{1}{x} \cdot \frac{1}{\frac{1}{x}} = 1$.
\begin{flalign*}
        \frac{1}{x} \cdot \frac{1}{\frac{1}{x}} &= 1         \text{ (by 1.12 (M5))}     &\\
x \cdot \frac{1}{x} \cdot \frac{1}{\frac{1}{x}} &= x \cdot 1 \text{ (by 1.12 (M5))}     &\\
                  1 \cdot \frac{1}{\frac{1}{x}} &= x         \text{ (by 1.12 (M4, M5))} &\\
                          \frac{1}{\frac{1}{x}} &= x         \text{ (by 1.12 (M4))}     
\end{flalign*}
