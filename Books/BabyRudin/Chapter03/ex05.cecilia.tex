\subsection*{Exercise 05 (Cecilia)}
We will start with a simple lemma.

\subsubsection*{Lemma 1}
If $ \limsup a_n = A $, $ -\infty < A < +\infty $ , then there is only finitely many points in the sequence $ a_n $ that are greater than $ A + \epsilon $ for any $ \epsilon > 0 $.

\subsubsection*{Proof}
Suppose that there are infinitely many points in the sequence $ a_n $ that are greater than $ A + \epsilon $ for some $ \epsilon > 0 $. These points must have a convergent subsequence (by Bolzano Weierstrass). For that subsequence, suppose $ \lim\limits_{k \to \infty} a_{n_k} = L \le A $, then there must be infinitely points in $ a_{n_k} $ such that $ |a_{n_k} - L| < \epsilon $, this contradicts with $ a_{n_k} > A + \epsilon $, so it must be the case that $ \lim\limits_{k \to \infty} a_{n_k} > A $, contradicting $ \limsup a_n = A $.

\subsubsection*{Lemma 2}
For a real sequence $ a_n $, if the set $ \{ a \in a_n | a > d \} $ is finite, then it is impossible for $ \lim\limits_{n \to \infty} a_n > d $.

\subsubsection*{Proof}
Suppose $ \lim\limits_{n \to \infty} a_n = d + e $ for $ e > 0 $, then there exists $ N $ such that for all $ n \ge N $, $ |a_n - (d + e)| > \frac{e}{2} $, but that would mean the set $ \{ a \in a_n | a > d \} $ is infinite.

\subsubsection*{Unbounded above cases}
Now, we will prove the main result. Suppose that $ \limsup a_n = A $ and $ \limsup b_n = B $. We will show that $ \limsup (a_n + b_n) \leq A + B $.

If $ \limsup a_n = +\infty $ and $ \limsup b_n \ne -\infty $, then there is nothing to prove because $ \limsup a_n + \limsup b_n = +\infty $ and any sequence $ c_n $ satisfies $ \limsup c_n \le +\infty $, including $ c_n = a_n + b_n $.

Similar arguments handles the case $ \limsup a_n \ne -\infty $ and $ \limsup b_n = +\infty $ as well.

\subsubsection*{Unbounded below cases}
Now suppose $ \limsup a_n = -\infty $ and $ \limsup b_n \ne +\infty $. Now every subsequences of $ a_n $ diverge to $ -\infty $ and $ b_n $ is bounded above, so any subsequence of $ a_{n_k} + b_{n_k} < a_{n_k} + B $ will still diverges to $ -\infty $, so
$  \limsup (a_n + b_n) \leq -\infty = \limsup a_n + \limsup b_n $.

Similar arguments handles the case $ \limsup a_n \ne +\infty $ and $ \limsup b_n = -\infty $ as well.

\subsubsection*{Finite Case}
Now suppose neither $ \limsup a_n $ nor $ \limsup b_n $ is $ \pm\infty $, we wanted to show that $ \limsup (a_n + b_n) \leq \limsup a_n + \limsup b_n $.

By the lemma, for any $ \epsilon > 0 $, there are only finitely many points in the sequence $ a_n $ that are greater than $ \limsup a_n + \frac{\epsilon}{2} $ , and there are only finitely many points in the sequence $ b_n $ that are greater than $ \limsup b_n + \frac{\epsilon}{2} $. This implies that there are only finitely many points in the sequence $ a_n + b_n $ that are greater than $ \limsup a_n + \limsup b_n + \epsilon $. There cannot be a subsequence in $ a_n + b_n $ that converges to a value larger than $ \limsup a_n + \limsup b_n + \epsilon $. Since $ \epsilon $ is arbitrary, $ \limsup (a_n + b_n) \leq \limsup a_n + \limsup b_n $.