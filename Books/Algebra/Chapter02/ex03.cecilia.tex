\subsection*{Exercise 2.3 (Cecilia)}

\subsubsection*{Part a}
For any $ x, y \in G $, we have

\begin{align*}
   & xy \\
  =& \sigma(x^{-1}) \sigma(y^{-1}) \\
  =& \sigma(x^{-1}y^{-1}) \\
  =& \sigma((yx)^{-1}) \\
  =& yx \\
\end{align*}

Therefore the group is Abelian.

\subsubsection*{Part b}

Work on the hint first. Suppose $ x^{-1}\sigma(x) = y^{-1}\sigma(y) $ we have:

\begin{align*}
   y^{-1}\sigma(y) &= x^{-1}\sigma(x) \\
           xy^{-1} &= \sigma(x)\sigma(y)^{-1} \\
           xy^{-1} &= \sigma(xy^{-1})
\end{align*}

As required, the only value such that $ g = \sigma(g) $ is 1, so $ xy^{-1} = 1 $, in other words, $ x = y $, the mapping is injective.

As $ G $ is finite, an injective map between two identical finite sets must be bijective. 

This is where I am right now.

\subsubsection*{Ideas on part b}

$ a $ can be represented as $ p^{-1} \sigma(p) $

$ b $ can be represented as $ q^{-1} \sigma(q) $

$ ab = p^{-1} \sigma(p) q^{-1} \sigma(q) = r^{-1} \sigma(r) $

$ ba = q^{-1} \sigma(q) p^{-1} \sigma(p) = s^{-1} \sigma(s) $

Why should they be equal?

It is natural to think it might involve part a. Maybe $ \sigma(x) $ must be $ x^{-1} $, maybe we can prove $ f(x) = x^{-1} $ is another isomorphism?
