\subsection*{Exercise 2.3 (Cecilia)}

\subsubsection*{Part a}
For any $ x, y \in G $, we have

\begin{align*}
   & xy \\
  =& \sigma(x^{-1}) \sigma(y^{-1}) \\
  =& \sigma(x^{-1}y^{-1})          \\
  =& \sigma((yx)^{-1})             \\
  =& yx \\
\end{align*}

Therefore the group is Abelian.

\subsubsection*{Part b}

Work on the hint first. Suppose $ x^{-1}\sigma(x) = y^{-1}\sigma(y) $ we have:

\begin{align*}
   y^{-1}\sigma(y) &= x^{-1}\sigma(x)         \\
           xy^{-1} &= \sigma(x)\sigma(y)^{-1} \\
           xy^{-1} &= \sigma(xy^{-1})
\end{align*}

As required, the only value such that $ g = \sigma(g) $ is 1, so $ xy^{-1} = 1 $, in other words, $ x = y $, the mapping is injective.

Now that we have shown every element in $ G $ can be written as . Consider this:

\begin{align*}
   & \sigma(x^{-1}\sigma(x))^{-1}           \\
  =& (\sigma(x^{-1})\sigma(\sigma(x)))^{-1} \\
  =& (\sigma(x)^{-1}\sigma(\sigma(x)))^{-1} \\
  =& (\sigma(x)^{-1}x)^{-1}                 \\
  =& x^{-1}\sigma(x)
\end{align*}

Because every element in $ G $ can be written as $ x^{-1}\sigma(x) $, we have $ \sigma(y)^{-1} = y $ for all $ y $ in $ G $, in other words, $ \sigma(y) = y^{-1} $. Therefore the group is Abelian by part (a).