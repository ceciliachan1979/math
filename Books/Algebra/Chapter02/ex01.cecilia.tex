\subsection*{Exercise 2.1 (Cecilia)}

Suppose $ H \ne G $ and $ K \ne G $, then there exists $ a,b \in G $ but $ b \notin H $ and $ a \notin K $. 

Because $ a \in G = H \cup K $ and $ a \notin K $, therefore $ a \in H $. Similarly, $ b \in K $.

Because $ H $ is a subgroup $ a \in H \implies a^{-1} \in H $. Similarly, $ b^{-1} \in K $.

$ ab \in G $. If $ ab \in H $, then $ a^{-1} ab = b \in H $, which is a contradiction.

Similarly, $ ab \in K $, then $ abb^{-1} = a \in K $, another contradiction.

So $ ab \notin H $ and also $ ab \notin K $, but $ ab \in G = H \cup K $, that is a contradiction as well.

Therefore it is impossible for $ H \ne G $ and $ K \ne G $. In other words, either $ H = G $ or $ K = G $.