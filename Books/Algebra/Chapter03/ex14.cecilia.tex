\subsection*{Exercise 3.14 (Cecilia)}

We will start with proving the existence of the subgroup with prime power order by induction on the power.

In particular, the statement to prove is that if a finite Abelian group has order $ n $ divisible by $ p^k $, then there exist a subgroup of $ g $ with order $ p^k$.

For the base case, when $ k = 1 $, we have problem 3.13 giving us an element $ g $ of order $ p $, which means the cyclic subgroup generated by $ g $ is the required subgroup.

For the induction case, if $ |G| $ is divisible by $ p^{k+1} $, by the induction hypothesis, we have a subgroup $ H $ of order $ p^k $. Again, $ H $ is normal in $ G $ because $ G $ is Abelian and therefore we can consider $ Q = G/H $.

Now $ |Q| $ has a factor of $ p $ and therefore $ Q $ has an element of order $ p $, $ (rH)^p = H $, $ r^p \in H $.

The subgroup generated by $ H $ and $ r $ is of order $ p^{k+1} $.

To show that, consider all products formed by elements in $ H $ and $ r $, using commutative property, we can move all factors in $ H $ to the left and lump them into a single element in $ H $, and the rest will be powers of $ r $, where using $ r^p \in H $, we can make sure the power is less than $ 0 \le power < p $.

The scheme show that every element can be written uniquely as a pair of an element in $ H $ and an element in $ \{r^0, r^1, \cdots r^{p-1}\} $, that establish the size of the generated subgroup to be $ p^{k+1} $.

Last but not least, after have a subgroup for all prime power factors. For arbitrary $ m $ that can be factorized into prime powers, the required subgroup for $ m $ is the internal direct product of these subgroups. Note that these subgroups are of different prime powers so that their elements orders can never be exactly equal unless it is the identity, this is why it is valid to make internal direct product for them.