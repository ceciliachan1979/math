\subsection*{Lang Exercise 1 (Cecilia)}

First, there is no group of order 0 or below, any group must have at least the identity element $e$.

For group with order 1, the only element is $e$, and therefore obviously Abelian.

For group $G$ with order higher than 1, we have at least one element $g$ that is not $e$. Consider the cyclic subgroup $C$ generated by $g$.

For order $p$ where $p$ is prime, $C$ must be $G$ because $C$ has at least two elements $g$ and $e$ but Lagrange's theorem implies the only subgroup of size at least 2 can only be the whole group, these cyclic groups are Abelian.

That leaves us with group of order 4. If $C$ is $G$, we are done. Otherwise, we have $C$ has order 2.

Besides $e$ and $g$, the other element $h$ must also generate a cyclic subgroup $D$. Again, if $D$ is $G$, we are done. Therefore $D$ has order 2 as well.

The rest is just fill in the blanks in the multiplication table by leveraging the fact that every row and column must have distinct values.

\[
\begin{array}{c|cc}
\text{Partial Table} & \qquad & \text{Filled Table} \\
\begin{array}{c|cccc}
     & e & g & h & i \\
    \hline
e & e & g & h & i \\
g & g & e &   &   \\
h & h &   & e &   \\
i & i &   &   &   \\
\end{array}
& \qquad &
\begin{array}{c|cccc}
     & e & g & h & i \\
    \hline
e & e & g & h & i \\
g & g & e & i & h \\
h & h & i & e & g \\
i & i & h & g & e \\
\end{array}
\end{array}
\]

Note that the filled table is symmetric, that proves it is Abelian.