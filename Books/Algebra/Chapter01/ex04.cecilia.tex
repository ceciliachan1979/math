\subsection*{Exercise 1.4 (Cecilia)}

\subsubsection*{Part a}
Checking that these defintion works out is routine.

\begin{table}[h]
\centering
\begin{minipage}{0.3\linewidth}
\centering
\begin{tabular}{c|c}
\textbf{Input} & \textbf{Output} \\ \hline
$a$ & $a$ \\
$b$ & $a$ \\
$c$ & $c$
\end{tabular}
\caption{Function \(e\) mappings}
\end{minipage}
\hfill
\begin{minipage}{0.3\linewidth}
\centering
\begin{tabular}{c|c}
\textbf{Input} & \textbf{Output} \\ \hline
$a$ & $c$ \\
$b$ & $c$ \\
$c$ & $a$
\end{tabular}
\caption{Function \(g\) mappings}
\end{minipage}
\hfill
\begin{minipage}{0.3\linewidth}
\centering
\begin{tabular}{c|cc}
$\times$ & $e$ & $g$ \\ \hline
$e$ & $e$ & $g$ \\
$g$ & $g$ & $e$ \\
\end{tabular}
\caption{Multiplication table for $e$ and $g$}
\end{minipage}
\end{table}

\subsubsection*{Part b}

We will first prove that the identity $e$ of a group with an injective mapping $f$ must be the identity function. Assume the contrary, then there exists $g$ such that $ e(g) = h \ne g$.

\begin{align*}
     & ef [g]  \\
    =& f(e(g)) \\
    =& f(h)    \\
  \ne& f(g)    && \text{(} f \text{ is injective)}
\end{align*}

Then we conclude $ef \ne f$ is a contradiction. $e$ must be the identity function.

With that, every element in $G$ must have an inverse that when composed produced the identity function. Now we conclude $ G \subseteq \text{Sym}(X) $.
