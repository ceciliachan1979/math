\subsection*{Exercise 1.9 (Cecilia)}

The set of unit quaternions 
\[
\{\pm 1, \pm i, \pm j, \pm k\}
\]
forms a group under quaternion multiplication. This can be seen by mapping each element to the corresponding complex $2 \times 2$ matrices:
\[
\begin{aligned}
1 &\mapsto \begin{bmatrix} 1 & 0 \\ 0 & 1 \end{bmatrix}, \quad
i \mapsto \begin{bmatrix} i & 0 \\ 0 & -i \end{bmatrix}, \quad
j \mapsto \begin{bmatrix} 0 & 1 \\ -1 & 0 \end{bmatrix}, \quad
k \mapsto \begin{bmatrix} 0 & i \\ i & 0 \end{bmatrix},
\end{aligned}
\]
where multiplication corresponds to matrix multiplication. The verification of the two-element multiplication identities (such as $ij = k$, $i^2 = j^2 = k^2 = -1$) is routine and thus omitted. Associativity follows directly from associativity of matrix multiplication, while the existence of inverses is immediate from these identities, as each element has a well-defined multiplicative inverse within the set.
