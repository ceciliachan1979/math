\subsection*{Exercise 1.2 (Cecilia)}

To start with, we prove that $R$ is a permutation group by showing it satisfy all the conditions listed in definition 1.3

We show that $R \subseteq \text{Sym}(G)$

For every $ r_x \in R$, we have:

\begin{align*}
   & r_x r_{x^{-1}} [t] \\
  =& txx^{-1}           \\
  =& t
\end{align*}

and also

\begin{align*}
   & r_{x^{-1}} r_x [t] \\
  =& tx^{-1}x           \\
  =& t
\end{align*}

Therefore $r_{x^{-1}}$ is the inverse of $r_x$ and therefore $r_x$ is a bijection on $G$ for all $x$.

With that. we also proved that for all $r_x$ in $R$, $(r_x)^{-1} = r_{x^{-1}} $ exists in $R$.

Now considering function composition, we have:

\begin{align*}
   & r_x r_y [t] \\
  =& (tx)y       \\
  =& t(xy)       \\
  =& r_{xy} [t]
\end{align*}

Therefore $R$ is closed under function composition. 

With all these, we conclude $ R $ is a permutation group.

Then, we will show that the map $f: x \mapsto r_x$ is an isomorphism from $G$ to $R$.

$f$ is a homomorphism because for any $t$ in $G$, we have

\begin{align*}
   & f(xy)[t]       \\
  =& r_{xy}[t]      \\
  =& txy            \\
  =& (tx)y          \\
  =& r_y [r_x[t]]   \\
  =& (r_x r_y)[t]   \\
  =& (f(x) f(y))[t]
\end{align*}

To show $f$ is a bijection, we define $f^{-1}: r_x \mapsto x$.

It is obvious that $f^{-1}$ is both a left and right inverse, so $f$ is both a homomorphism and a bijection, and therefore an isomorphism.

For $L$, everything is similar.

We prove that $L$ is a permutation group by showing it satisfy all the conditions listed in definition 1.3

We show that $L \subseteq \text{Sym}(G)$

For every $ l_x \in L$, we have:

\begin{align*}
   & l_x l_{x^{-1}} [t] \\
  =& x^{-1}xt           \\
  =& t
\end{align*}

and also

\begin{align*}
   & l_{x^{-1}} l_x [t] \\
  =& xx^{-1}t           \\
  =& t
\end{align*}

Therefore $l_{x^{-1}}$ is the inverse of $l_x$ and therefore $l_x$ is a bijection on $G$ for all $x$.

With that. we also proved that for all $l_x$ in $L$, $(l_x)^{-1} = l_{x^{-1}} $ exists in $L$.

Now considering function composition, we have:

\begin{align*}
   & l_x l_y [t] \\
  =& y(xt)       \\
  =& (yx)t       \\
  =& l_{yx} [t]
\end{align*}

Therefore $L$ is closed under function composition. 

With all these, we conclude $ L $ is a permutation group.

Then, we will show that the map $g: x \mapsto l_{x^{-1}}$ is an isomorphism from $G$ to $L$.

$g$ is a homomorphism because for any $t$ in $G$, we have

\begin{align*}
   & g(xy)[t]                   \\
  =& l_{(xy)^{-1}}[t]           \\
  =& (xy)^{-1}t                 \\
  =& y^{-1}x^{-1}t              \\
  =& l_{y^{-1}}[l_{x^{-1}}[t]]  \\
  =& (l_{x^{-1}} l_{y^{-1}})[t] \\
  =& (g(x) g(y))[t]             \\
\end{align*}

To show $g$ is a bijection, we define $g^{-1}: l_x \mapsto x^{-1}$.

It is obvious that $g^{-1}$ is both a left and right inverse, so $g$ is both a homomorphism and a bijection, and therefore an isomorphism.
