\subsection*{Exercise 1.1 (Cecilia)}

Suppose the group is transitive, Abelian but not regular, for some $x, y$ in $X$, there exists $g_1 \ne g_2$ such that $g_1(x) = y = g_2(x) $.

For any $p$ in $X$, by transitivity, there exists $f$ such that $f(x) = p$

\begin{align*}
g_1(p) 
&= g_1(f(x))      && \text{(since } f(x) = p \text{)} \\
&= f(g_1(x))      && \text{(Abelian property)} \\
&= f(y)          && \text{(assumption } g_1(x) = y \text{)} \\
&= f(g_2(x))      && \text{(assumption } g_2(x) = y \text{)} \\
&= g_2(f(x))      && \text{(Abelian property)} \\
&= g_2(p)         && \text{(since } f(x) = p \text{)} \,.
\end{align*}

Therefore $g_1 = g_2$ for all $p$ in $X$ but $g_1 \ne g_2$, that's a contradiction, and so the group must be regular.