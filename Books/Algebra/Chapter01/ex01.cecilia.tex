\subsection*{Exercise 1.1 (Cecilia)}

Suppose the group is transitive, Abelian but not regular, for some $x, y$ in $X$, there exists $g_1 \ne g_2$ such that $g_1(x) = y = g_2(x) $.

For any $p$ in $X$, by transitivity, there exists $f$ such that $f(x) = p$

\begin{align*}
g_1(p) 
&= g_1(f(x))      && \text{(since } f(x) = p \text{)} \\
&= f(g_1(x))      && \text{(Abelian property)} \\
&= f(y)          && \text{(assumption } g_1(x) = y \text{)} \\
&= f(g_2(x))      && \text{(assumption } g_2(x) = y \text{)} \\
&= g_2(f(x))      && \text{(Abelian property)} \\
&= g_2(p)         && \text{(since } f(x) = p \text{)} \,.
\end{align*}

Therefore $g_1 = g_2$ for all $p$ in $X$ but $g_1 \ne g_2$, that's a contradiction, and so the group must be regular.

% 
% The map
% 
% f: x -> rx is the iso from G to R
% 
% f is homo because f(xy)[t] = rxy[t] = txy = (tx)y = ry [rx[t]] = (rx ry)[t] = (f(x) f(y))[t]
% 
% define f inverse as rx -> x. 
% 
% It is both left and right inverse, therefore f is bijective.
% 
% g: x -> lx^{-1} is the iso from G to L
% 
% g is homo because g(xy)[t] = l(xy)^{-1}[t] = (xy)^{-1}t = y^{-1}x^{-1}t = ly^{-1}[lx^{-1}[t]] = (lx_{-1} ly_{-1})[t] = (g(x) g(y))[t]
% 
% define g inverse as lx -> x^{-1}. 
% 
% It is both left and right inverse, therefore f is bijective.
% 
% ================================================================================================
% 
% Suppose f is in L, i.e. f = lx
% 
% for any r = ry
% 
% fr[t] = f[r[t]] = x(gy) = xgy
% 
% rf[t] = r[f[t]] = (xg)y = xgy
% 
% Therefore L superset given set
% 
% Suppose f in given set, for any t, consider r = rt inverse
% 
% fr = rf
% fr[t] = rf[t]
% r[f[t]] = f[r[t]]
% f[t] t^{-1} = f[t t^{-1}]
% f[t] t^{-1} = f[e]          // e is identity
% f[t] = t f[e]
% 
% Therefore f = l_f[e]
% 
% Therefore given set superset L
% 
% ================================================================================================
% 
% 