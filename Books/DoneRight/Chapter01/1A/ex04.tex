\subsubsection*{Exercise 04}

\begin{flushleft}
Let $\alpha$, $\beta$, and $\lambda$ be three complex numbers such that $\alpha = p + iq$, $\beta  = m + in$, and $\lambda = u + iv$, where $p,\ q,\ m,\ n,\ u,\ v \in \mathbb{R}$. Then, we can express the product of $\alpha$, $\beta$, and $\lambda$ as follows:
\begin{align*}
\lambda \cdot (\alpha + \beta) &= (u + iv) \cdot (p + iq + m + in) \\
                               &= (u + iv) \cdot ((p + m) + i(q + n)) \\
                               &= u(p + m) + iu(q + n) + iv(p + m) - v(q + n) \\
                               &= (u(p + m) - v(q + n)) + i(u(q + n) + v(p + m)) \\
                               &= Left \\ 
\lambda \cdot \alpha + \lambda  \cdot \beta &= (u + iv) \cdot (p + iq) + (u + iv) \cdot (m + in) \\
                                            &= (up + iqu + ivp - vp) + (um + inu + ivm - vn) \\
                                            &= (up - vp + um - vn) + i(qu + vp + nu + vm) \\
                                            &= (up + um - vp - vn) + i(qu + nu + vp + vm) \\
                                            &= (u(p + m) - v(q + n)) + i(u(q + n) + v(p + m)) \\
                                            &= Right
\end{align*}
Since Left = Right, we can conclude that $\lambda \cdot (\alpha + \beta) = \lambda \cdot \alpha + \lambda  \cdot \beta$.
\end{flushleft}

