\subsubsection*{Exercise 03}

\begin{flushleft}
Let $\alpha$, $\beta$, and $\lambda$ be three complex numbers such that $\alpha = p + iq$, $\beta  = m + in$, and $\lambda = u + iv$, where $p,\ q,\ m,\ n,\ u,\ v \in \mathbb{R}$. Then, we can express the product of $\alpha$, $\beta$, and $\lambda$ as follows:
\begin{align*}
(\alpha \cdot \beta) \cdot \lambda &= ((p + iq) \cdot (m + in)) \cdot (u + iv)               \\
                                   &= ((pm - qn) + i(pn + mq)) \cdot (u + iv)                \\
                                   &= ((pm - qn)u - v(pn + mq)) + i((pm - qn)v + u(pn + mq)) \\
                                   &= (pmu - qnu - vpn - vmq) + i(pmv - qnv + upn + umq)     \\
\alpha \cdot (\beta \cdot \lambda) &= (p + iq) \cdot ((m + in) \cdot (u + iv))               \\
                                   &= (p + iq) \cdot ((mu - nv) + i(nu + mv))                \\
                                   &= p(mu - nv) - q(nu + mv) + i(p(nu + mv) + q(mu - nv))   \\
                                   &= (pmu - pnv - qnu - qmv) + i(pnu + pmv + qmu - qnv)     \\
                                   &= (pmu - vpn - qnu - vmq) + i(upn + pmv + umq - qnv)     \\
                                   &= (pmu - qnu - vpn - vmq) + i(pmv - qnv + upn + umq)     
\end{align*}
Since the two expressions are equal, we can conclude that multiplication is associative in the set of complex numbers $\mathbb{C}$.
\end{flushleft}
