\subsubsection*{Exercise 02 (Gapry)}

\begin{flushleft}
Let $\alpha$, $\beta$, and $\lambda$ be three complex numbers such that $\alpha = p + iq$, $\beta  = m + in$, and $\lambda = u + iv$, where $p,\ q,\ m,\ n,\ u,\ v \in \mathbb{R}$.

Then, we can express the sum of $\alpha$, $\beta$, and $\lambda$ as follows:
\begin{align*}
(\alpha + \beta) + \lambda &= ((p + iq) + (m + in)) + (u + iv) \\
                           &= (p + m) + i(q + n) + (u + iv)    \\
                           &= (p + m + u) + i(q + n + v)       \\
\alpha + (\beta + \lambda) &= (p + iq) + ((m + in) + (u + iv)) \\
                           &= (p + iq) + (m + u) + i(n + v)    \\
                           &= (p + m + u) + i(q + n + v)
\end{align*}
Since the two expressions are equal, we can conclude that addition is associative in the set of complex numbers $\mathbb{C}$.
\end{flushleft}