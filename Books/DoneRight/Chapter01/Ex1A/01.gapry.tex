\subsubsection*{Exercise 01 (Gapry)}

\begin{flushleft}
Let $\alpha$ and $\beta$ be two complex numbers such that $\alpha = p + iq$ and $\beta  = m + in$, where $p,\ q,\ m,\ n \in \mathbb{R}$. Then, we can express the sum of $\alpha$ and $\beta$ as follows:
\begin{align*}
\alpha + \beta &= (p + iq) + (m + in) \\
               &= (p + m) + i(q + n)
\end{align*}

Similarly, the sum of $\beta$ and $\alpha$ is:
\begin{align*}
\beta  + \alpha &= (m + in) + (p + iq) \\
                &= (m + p) + i(n + q)
\end{align*}

Since addition is commutative in the set of real numbers $\mathbb{R}$, we have $(p + m) = (m + p)$ and $(q + n) = (n + q)$. Therefore, we can conclude that addition is also commutative in the set of complex numbers $\mathbb{C}$.
\end{flushleft}