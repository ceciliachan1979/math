\subsubsection*{Exercise 04 (Gapry)}

\begin{flushleft}

Let \( V = \mathcal{L}(\mathbb{R}^5, \mathbb{R}^4) \) and
\(
U = \{ T \in \mathcal{L}(\mathbb{R}^5, \mathbb{R}^4) \mid \dim(\text{null } T) > 2 \}
\) \newline

Suppose \( W_1 \in V \) and \( W_1 \in U \), \( W_2 \in V \) and \( W_2 \in U \), then it follows 
\[
\dim(\text{null } W_1) > 2 \quad \text{and} \quad \dim(\text{null } W_2) > 2
\]

By $\text{Fundamental Theorem of Linear Maps}$, we know 
\[
\dim(\text{range } W_1) <= 2 \quad \text{and} \quad \dim(\text{range } W_2) <= 2
\]

Again, by $\text{Fundamental Theorem of Linear Maps}$, we know 
\[
\dim(W_1 + W_2) = \dim(\text{null } (W_1 + W_2)) + \dim(\text{range } (W_1 + W_2)) = 5 
\]

Also, we know the 
\[
\dim(W_1 + W_2) = \dim W_1 + \dim W_2 - \dim(W_1 \cap W_2)
\]

it follows 
\[
\dim(W_1 + W_2) <= \dim W_1 + \dim W_2
\]

Hence, we know 
\[
\dim(\text{range } (W_1 + W_2)) <= \dim(\text{range } (W_1)) + \dim(\text{range } (W_2)) <= 4
\]

That means 
\[
\dim(\text{null } (W_1 + W_2)) >= 1
\]

If $U$ is the subspace of $V$, 
\[
\dim(\text{null } (W_1 + W_2)) > 2
\]

But we prove 
\[
\dim(\text{null } (W_1 + W_2)) >= 1
\]
hence $U$ isn't the subspace of $V$.

\end{flushleft}