\subsubsection*{Exercise 1.1.7. (Gapry)}

\begin{flushleft}

Let's define the nonstandard ordering of the set of natural numbers $\mathbb{N}$.
\begin{equation*}
  n \prec m =
    \begin{cases}
      n < m & \text{if (n mod 3) $<$ (m mod 3)} \\
      n = m & \text{if (n mod 3) $=$ (m mod 3)} \\
      n > m & \text{if (n mod 3) $>$ (m mod 3)}
    \end{cases}       
\end{equation*}

Let's define the subset $A$ of natural numbers $\mathbb{N}$ with the order $n \prec m$. \\
$A = \{n \mod 7 == 0, n \in \mathbb{Z^+}\} \subset \mathbb{N}$.
\vspace{10px}

Let's define $b = 7n + 1, n \in \mathbb{Z^+}$
\vspace{10px}

Since $b \notin A$, $b \in \mathbb{Z^+}$ and $b$ is an upper bound of $A$ \\
Let's analyze all upper bounds of $A$,
$\{7n + 1, 7n + 2, 7n + 3, 7n + 4, 7n + 5, 7n + 6, n \in Z^{+} \}$,
it's obvious $7n + 1$ is the minimum of them, hence $7n + 1$ is the $\sup A$. Q.E.D.

\end{flushleft}