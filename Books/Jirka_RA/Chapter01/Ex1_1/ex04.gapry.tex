\subsubsection*{Exercise 1.1.4 (Gapry)}

\begin{flushleft}
Since we know $\inf A \le \sup A$ is true, let's analyze $\inf B \le \inf A$ and $\sup A \le \sup B$.

\textbf{Part A: $\inf B \le \inf A$} \\
Proof by contradiction: 
Given that $A \subset B$, it follows that the lower bounds of $B$ are also the lower bounds of $A$. 
Therefore, $\inf B$ is a lower bound of $A$. If $\inf B > \inf A$, 
this implies that there exists a lower bound of $A$ greater than the greatest lower bound of $A$, 
it contradicts.

\textbf{Part B: $\sup A \le \sup B$} \\
Proof by contradiction: 
Given that $A \subset B$, it follows that the upper bounds of $B$ are also the upper bounds of $A$. 
Therefore, $\sup B$ is an upper bound of $A$. If $\sup A > \sup B$, 
this implies that there exists a upper bound of $A$ less than the least upper bound of $A$, 
it contradicts.

\textbf{Conclusion:} \\
Since $\inf A \le \sup A$ and $\inf B \le \inf A$ and $\sup A \le \sup B$ are true, 
$\inf B \le \inf A \le \sup A \le \sup B$ is true. Q.E.D.
\end{flushleft}
