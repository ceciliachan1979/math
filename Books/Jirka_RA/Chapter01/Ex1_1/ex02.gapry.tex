\subsubsection*{Exercise 1.1.2. (Gapry)}

\begin{flushleft}
Since $\mathbb{A}$ is a subset of $\mathbb{S}$, $\mathbb{A}$ is ordered set. Arrange the elements x of A such that $\{ x_i \ | \ x_n \le x_{n + 1}, n \ge 1, n \in \mathbb{N} \}$
\end{flushleft}

\begin{flushleft}
Basis Case: 

If n = 1, then $A = \{x_1\}$, $\inf A = x_1$, $\sup A = x_1$

If n = 2, then $A = \{x_1, x_2\}$, $x_1 \le x_2$, $\inf A = x_1$, $\sup A = x_2$

If n = 3, then $A = \{x_1, x_2, x_3\}$, $x_1 \le x_2 \le x_3$, $\inf A = x_1$, $\sup A = x_3$
\end{flushleft}

\begin{flushleft}
Induction Hypothesis:

If $A = \{x_1, \cdots, x_n\}$, $x_1 \le \cdots \le x_n$, $\inf A = x_1$, $\sup A = x_n$ is true, then
$A = \{x_1, \cdots, x_n\} \cup \{ y \},\ y \in \mathbb{S} \setminus \{x_1, \cdots, x_n\}$ 

\textbf{Case 1:}
if $y \ge x_n$, then $\inf A = x_1$, $\sup A = y$

\textbf{Case 2:}
if $y \le x_1$, then $\inf A = y$, $\sup A = x_n$

\textbf{Case 3:}
if $x_1 < y < x_n$, then $\inf A = x_1$, $\sup A = x_n$
\end{flushleft}

\begin{flushleft}
By the principle of induction, we know that $\inf A$ and $\sup A$ exist and are in $\mathbb{A}$, hence it's bounded.
\end{flushleft}