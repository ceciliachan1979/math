\subsubsection*{Exercise 1.1.7. (Cecilia)}

\begin{flushleft}

Let's define the nonstandard ordering of the set of natural numbers $\mathbb{N}$.
\begin{equation*}
  n \prec m =
    \begin{cases}
      n < m        & \text{if $ n \ne 1 $ and $ m \ne 1 $} \\
      \text{true}  & \text{if $ n \ne 1 $ and $ m = 1 $} \\
      \text{false} & \text{if $ n = 1 $ }
    \end{cases}
\end{equation*}

As usual, $ n \succ m $ if $ n \ne m $ and not $ n \prec m $.

\textbf{Trichotomy} Claim: For any $ a, b \in \mathbb{N} $, exactly one of the following holds:
\begin{enumerate}
  \item{$ a \prec b $}
  \item{$ a = b $}
  \item{$ a \succ b $}
\end{enumerate}

Proof: In the case $ a \ne 1 $ and $ b \ne 1 $, trichotomy follows from the trichotomy of standard comparsion

Suppose $ a = 1 $, $ b \ne 1 $, then 
\begin{enumerate}
  \item{$ a \prec b $ is false, by definition}
  \item{$ a = b $ is false.}
  \item{$ a \succ b $ is true, by definition}
\end{enumerate}

Suppose $ a \ne 1 $, $ b = 1 $, then 
\begin{enumerate}
  \item{$ a \prec b $ is true, by definition}
  \item{$ a = b $ is false.}
  \item{$ a \succ b $ is false, by definition}
\end{enumerate}

Suppose $ a = 1 $, $ b = 1 $, then 
\begin{enumerate}
  \item{$ a \prec b $ is false, by definition}
  \item{$ a = b $ is true.}
  \item{$ a \succ b $ is false, by definition}
\end{enumerate}

So trichotomy holds for all cases.

\textbf{Transitivity} Claim: For any $ a, b, c \in \mathbb{N} $, if $ x \prec y $ and $ y \prec z $, then $ x \prec z $.

It is impossible for $ x = 1 $ or $ y = 1 $, suppose $ x \ne 1 $ and $ z = 1 $, we have $ x \prec z $ unconditionally.

Otherwise transitivity follows from the transitivity of standard comparison.

\textbf{Least upper bound example}
Consider the subset $ A = \{x \in \mathbb{N} | x > 1\} $, the subset is bounded above (under $ \prec $) by $ 1 $. We claim that the least upper bound is $ 1 $ and is therefore not belongs to the set $ A $.

Obviously $ 1 $ is an upper bound. For any number $ n \in \mathbb{N} $ such that $ n \prec 1 $. The number $ n + 1 \in A $, so it is not an upper bound. Therefore $ 1 $ is the least upper bound.

\end{flushleft}