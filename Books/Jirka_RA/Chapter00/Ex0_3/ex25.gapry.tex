\subsubsection*{Exercise 0.3.25. (Gapry)}

\begin{flushleft}
\textbf{Part (a)} \\
\vspace{10px}

\text{Case 1: if $|\mathbb{N}| = |\mathbb{S}|$}, according to \textbf{Definition 0.3.29}, $|\mathbb{S}|$ is countable infinite. Since $|\mathbb{S}|$ is countable infinite, for all subset $\mathbb{A} \subset \mathbb{S}$, $\mathbb{A}$ is also countable infinite. \\
\vspace{10px}

\text{Case 2: if $|\mathbb{N}| < |\mathbb{S}|$}, there exists a set $\mathbb{A} \subset \mathbb{N} \subset \mathbb{S}$, according to \textbf{Definition 0.3.29}, $\mathbb{A}$ is countable infinite. \\
\vspace{10px}

Therefore, there exists a countably infinite subset $\mathbb{A} \subset \mathbb{S}$. \\
\vspace{10px}

\textbf{Part (b)} \\
Let's define a function $f: (\mathbb{S \setminus A}) \rightarrow \mathbb{S}$. To prove that $f$ is bijection, we need demonstrate both injection and surjection. \\
\vspace{10px}

\text{Case: Injection} \\
Since the domain is $\mathbb{S \setminus A}$ and range $\mathbb{S}$, for all $y \in \mathbb{S}$, the $f^{-1}(y)$ is undefined in $\mathbb{A}$ or consists of a single element in $\mathbb{S \setminus A}$, hence $f$ is an injection. \\
\vspace{10px}

\text{Case: Surjection} \\
For all $x \in (\mathbb{S \setminus A})$, $f(x) \in \mathbb{A}$ or $f(x) \in (\mathbb{S \setminus A)}$, since the subset $\mathbb{A}$ is countably infinite, $\mathbb{S}$ is infinite and $(\mathbb{S \setminus A)} \cup \mathbb{A} = \mathbb{S}$, hence the range is equal codomain of f, thus $f$ is a surjection. \\
\vspace{10px}

Hence, there exists a bijection between $\mathbb{S \setminus A}$ and $\mathbb{S}$.
\end{flushleft}
