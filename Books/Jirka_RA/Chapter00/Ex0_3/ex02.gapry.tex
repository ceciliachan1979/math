\subsubsection*{Exercise 0.3.2. (Cecilia, Gapry)}

\begin{flushleft}
Introduction \\
\vspace{10px}

We can use the \textbf{Well Ordering Property of} $\mathbb{N}$ to prove that 
the \textbf{Principle of Strong Induction} is equivalent to the 
\textbf{Standard Induction}. First, we prove that the \textbf{Well Ordering 
Property of} $\mathbb{N}$ is equivalent to the \textbf{Standard Induction}.  
Second, we prove that the \textbf{Well Ordering Property of} $\mathbb{N}$ is 
equivalent to the \textbf{Principle of Strong Induction}. So we can conclude 
that the \textbf{Principle of Strong Induction} is equivalent to the 
\textbf{Standard Induction}.
\end{flushleft}

\begin{flushleft}
Part A: \textbf{Well Ordering Property of} $\mathbb{N}$ is equivalent to the 
\textbf{Standard Induction}. \\
\vspace{10px}

Prove: \textbf{Well Ordering Property of} $\mathbb{N}$ $\implies$ 
\textbf{Standard Induction} is true. \\
\vspace{10px}

Proof by Contradiction: If the \textbf{Well Ordering Property of} $\mathbb{N}$ 
is true, then the \textbf{Standard Induction} is false. It follows that there 
exists a statement $P(n)$ which is false if $P(1)$ is true and $P(k) \implies 
P(k + 1)$ is also true, where $n \in \mathbb{N}$, $k \in \mathbb{N}$. Let's 
define two sets, one is $S_F = \{ P(n) \text{ is false } |\ n \in \mathbb{N} 
\}$, the other is $S_T = \{ P(m) \text{ is true } |\ m \in \mathbb{N} \}$, 
$n \ne m$. According to \textbf{Well Ordering Property of} $\mathbb{N}$, we
know that $S_F$ exists a least index $k$ such that $P(k)$ is false, it follows 
$P(k - 1) \in S_T$ which $P(k - 1)$ is true since $k - 1$ is not the least 
index of $S_F$. For now, we know $P(k - 1) \implies P(k)$ is false which  
contradicts with our assumption $P(k) \implies P(k + 1)$ is true. Hence, we 
know that \textbf{Well Ordering Property of} $\mathbb{N}$ $\implies$ 
\textbf{Standard Induction} is true.
\end{flushleft}

\begin{flushleft}
Prove: \textbf{Standard Induction} $\implies$ \textbf{Well Ordering Property 
of} $\mathbb{N}$ is true. \\
\vspace{10px}

The contraposition of \textbf{Well Ordering Property of} $\mathbb{N}$ is
that if the least element doesn't exist, then $S$ is the empty set. 
Let's use standard induction to prove the contraposition is true. \\
\vspace{10px}

(i) Basis Statement: If $P(1)$ is true, 1 $\in$ S, then 1 is the least element 
of S. It follows that if 1 $\notin$ S, then $S$ is the empty set. It proves 
that P(1) is true if $S$ doesn't not have a least element, then $\{1\} \cap S$ 
is the empty set.\\
\vspace{10px}

(ii) Induction Step: 
Assume P(k) is true if $S$ doesn't not have a least element, then $\{1, \cdots, 
k\} \cap S$ is the empty set. For now, If $k + 1 \in S$, then it is the least 
element of $S$. It follows that if $k + 1 \notin S$, then $S$ is the empty set. 
It proves that P(k + 1) is true if $S$ doesn't not have a least element, then 
$\{1, \cdots, k, k + 1\} \cap S$ is the empty set.\\
\vspace{10px}
 
Since the contraposition of \textbf{Well Ordering Property of} $\mathbb{N}$ is 
true, the \textbf{Well Ordering Property of} $\mathbb{N}$ is also true. So, we 
know that \textbf{Standard Induction} $\implies$ \textbf{Well Ordering Property 
of} $\mathbb{N}$ is true. 
\end{flushleft}

\begin{flushleft}
Hence, we can conclude that \textbf{Well Ordering Property of} $\mathbb{N}$ is 
equivalent to the \textbf{Standard Induction}. \\
\end{flushleft}

\begin{flushleft}
Part B: \textbf{Well Ordering Property of} $\mathbb{N}$ is equivalent to the 
\textbf{Principle of Strong Induction}. \\
\vspace{10px}

Prove: \textbf{Well Ordering Property of} $\mathbb{N}$ $\implies$ 
\textbf{Principle of Strong Induction} is true. \\
\vspace{10px}

Proof by Contradiction: If the \textbf{Well Ordering Property of} $\mathbb{N}$ 
is true, then the \textbf{Strong Induction} is false. It follows that there 
exists a statement $P(n)$ is false if $P(1)$ is true and $P(1) \land \cdots 
\land P(k) \implies P(k + 1)$ is also true, $n \in \mathbb{N}$, $k \in 
\mathbb{N}$. Let's define two sets, one is $S_F = \{P(n) \text{ is false } |\ 
n \in \mathbb{N} \}$, the other is $S_T = \{ P(m) \text{ is true } |\ m \in 
\mathbb{N} \}$, $n \ne m$. According to \textbf{Well Ordering Property of} 
$\mathbb{N}$, we know $S_F$ exists a least index $k$ such that $P(k)$ is false, 
it follows that $P(1), \cdots, P(k - 1)$ are all in $S_T$ since $1, \cdots, 
k - 1$ are all less than $k$. For now, we know that $P(1) \land \cdots \land 
P(k - 1)$ is true,  $P(k)$ is false. It follows that $P(1) \land \cdots \land 
P(k - 1) \implies P(k)$ is false which contradicts with our assumption $P(1) 
\land \cdots \land P(k) \implies P(k + 1)$ is true. Hence, we know \textbf{Well 
Ordering Property of} $\mathbb{N}$ $\implies$ \textbf{Principle of Strong 
Induction} is true.
\end{flushleft}

\begin{flushleft}
Prove: \textbf{Principle of Strong Induction} $\implies$ \textbf{Well Ordering 
Property of} $\mathbb{N}$ is true. \\
\vspace{10px}

The contraposition of \textbf{Well Ordering Property of} $\mathbb{N}$ is that 
if the least element doesn't exist, then $S$ is the empty set. Let's use strong 
induction to prove the contraposition is true. \\
\vspace{10px}

(i) Basis Statement: 
If $P(1)$ is true, 1 $\in$ S, then 1 is the least element of S. It follows that 
if 1 $\notin$ S, then $S$ is the empty set. \\
\vspace{10px}

(ii) Induction Step: 
If $P(1), \cdots, P(k)$ are all true and $1, \cdots, k \in S$, then 1 is the 
least element of S. It follows that if $1, \cdots, k$ are both not in $S$, 
then $S$ is the empty set. For now, If $k + 1 \in S$, then it is the least 
element of $S$. It follows that if $k + 1 \notin S$, then $S$ is the 
empty set. \\
\vspace{10px}

Since the contraposition of \textbf{Well Ordering Property of} $\mathbb{N}$ is 
true, the \textbf{Well Ordering Property of} $\mathbb{N}$ is also true. Hence, 
\textbf{Principle of Strong Induction} $\implies$ \textbf{Well Ordering 
Property of} $\mathbb{N}$ is true.
\end{flushleft}

\begin{flushleft}
Hence, we can conclude that \textbf{Well Ordering Property of} $\mathbb{N}$ is 
equivalent to the \textbf{Principle of Strong Induction}. \\
\end{flushleft}

\begin{flushleft}
Summary \\
\vspace{10px}

According to the results of Part A and Part B, we know that the 
\textbf{Principle of Strong Induction} is equivalent to the \textbf{Standard 
Induction} is true. Q.E.D.
\end{flushleft}
