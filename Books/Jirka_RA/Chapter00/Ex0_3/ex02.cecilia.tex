\subsubsection*{Exercise 0.3.2. (Cecilia)}

\begin{flushleft}
To show that the principle of standard induction and the principle of strong induction is equivalent, we show that a proof written in one form can be transformed into another. 
\vspace{10px}

If we had a proof written in standard induction. The proof will look like this.
\vspace{10px}

A proof of $S(1)$,
A proof of $S(k) \implies S(k + 1)$ for all positive integers $k$.
And then we claim $S(k)$ is true for all positive integers $k$ by standard induction.
\vspace{10px}

Such a proof can be trivially transformed into a proof using strong induction as follows, 
the second line only assumes more.
\vspace{10px}

A proof of $S(1)$,
A proof of $S(1), \cdots, S(k) \implies S(k + 1)$ for all positive integers $k$.
And then we claim $S(k)$ is true for all positive integers k by strong induction.
\vspace{10px}

On the other hand, if we had a proof written in strong induction. 
The proof will look like this.
\vspace{10px}

A proof of $S(1)$
A proof of $S(1), \cdots, S(k) \implies S(k + 1)$ for all positive integers $k$.
And then we claim S(k) is true for all positive integers k by strong induction.
\vspace{10px}

Such a proof can be transformed into a proof using standard induction as follow. 
\vspace{10px}

Let $T(k)$ to be the statement $S(n)$ is true for all $n$ in $[1, k]$
\vspace{10px}

A proof of $T(1)$ is simply a proof of $S(1)$
A proof of $T(k) \implies T(k + 1)$ is simply a proof 
$S(1), \cdots, S(k) \implies S(1), \cdots, S(k), S(k + 1)$, 
we can proof trivially that $S(1), \cdots, S(k) \implies S(1), \cdots, S(k)$, 
and we already had a proof that $S(1), \cdots, S(k) \implies S(k + 1)$
And then we claim $T(k)$ is true for all positive integers $k$ by standard induction, 
such a claim obviously also imply $S(k)$ is true for all positive integer $k$.
\vspace{10px}

Since a proof written in either form can be transformed into each other, 
the two methods are equivalent. Q.E.D.
\vspace{10px}

Here is the \textbf{counterexample}:
\vspace{10px}

\textbf{Standard Induction} \\
Proof: $1 + 2 + \cdots + n = \frac{n(n + 1)}{2}$ for all positive integer $n$
\vspace{10px}

Let $S(k)$ be the statement $1 + \cdots + k = \frac{k(k + 1)}{2}$
\vspace{10px}

Base case: 
$S(1)$ is true because $1 + \cdots + 1 = 1 = \frac{1(1 + 1)}{2}$
\vspace{10px}

Assume $S(k)$ is true
$S(k + 1)$ is true because $1 + \cdots + k + k + 1 
                         = \frac{k(k + 1)}{2} + (k + 1) 
                         = \frac{(k+1)(k + 2)}{2}$ \\
Therefore by the principle of standard mathematical induction, 
the statement $S(n)$ is true for all positive integers $n$.
\vspace{10px}

\textbf{Transformed Strong Induction} \\
Proof: $1 + 2 + \cdots + n = \frac{n(n + 1)}{2}$ for all positive integer $n$
\vspace{10px}

Let $S(k)$ be the statement $1 + \cdots + k = \frac{k(k + 1)}{2}$
\vspace{10px}

Base case: 
$S(1)$ is true because $1 + \cdots + 1 = 1 = \frac{1(1 + 1)}{2}$
\vspace{10px}

Assume $S(1), \cdots, S(k)$ is true
$S(k + 1)$ is true because $1 + \cdots + k + k + 1 
                         = \frac{k(k + 1)}{2} + (k + 1) 
                         = \frac{(k + 1)(k + 2)}{2}$ \\
Therefore by the principle of strong mathematical induction, 
the statement $S(n)$ is true for all positive integers $n$.
\end{flushleft}