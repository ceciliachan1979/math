\subsubsection*{Exercise 0.3.13. (Gapry)}

\begin{flushleft}
Basis Case, If n = 1, then $\frac{1}{1 \cdot (1 + 1)} = \frac{1}{1 + 1} = \frac{1}{2}$
\end{flushleft}

\begin{flushleft}
Induction Hypothesis, If $\frac{1}{1 \cdot 2} + \frac{1}{2 \cdot 3} + \cdots + \frac{1}{n \cdot (n + 1)} = \frac{n}{n + 1}$ is true, then 
\begin{flalign*}
\frac{1}{1 \cdot 2} + 
\frac{1}{2 \cdot 3} + 
\cdots + 
\frac{1}{n \cdot (n + 1)} + 
\frac{1}{(n + 1) \cdot (n + 2)} &= \frac{n}{n + 1} + \frac{1}{(n + 1) \cdot (n + 2)} &\\
                                &= \frac{n(n + 2)}{(n + 1)(n + 2)} + 
                                   \frac{1}{(n + 1) \cdot (n + 2)}     &\\
                                &= \frac{n(n + 2) + 1}{(n + 1)(n + 2)} &\\
                                &= \frac{n^2 + 2n + 1}{(n + 1)(n + 2)} &\\
                                &= \frac{(n + 1)^2}{(n + 1)(n + 2)}    &\\
                                &= \frac{(n + 1)^{\bcancel{2}}}{\bcancel{(n + 1)}(n + 2)} &\\
                                &= \frac{n + 1}{n + 2} 
\end{flalign*}
By the principle of induction, $\frac{1}{1 \cdot 2} + \frac{1}{2 \cdot 3} + \cdots + \frac{1}{n \cdot (n + 1)} = \frac{n}{n + 1}$ is true for all $n \in \mathbf{N}$.
\end{flushleft}