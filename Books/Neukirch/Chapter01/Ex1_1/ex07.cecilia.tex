\subsection*{Exercise 7 (Cecilia)}
To prove that $ \mathbb{Z}[\sqrt{2}] $ is Euclidean, we will divide two arbitrary values in $ \mathbb{Z}[\sqrt{2}] $ and show that the remainder have a smaller norm given by our chosen norm function. 

The division algorithm is simple, the quoient is chosen to be the rounded result of the usual division is $ \mathbb{Q}[\sqrt{2}] $, the norm function is chosen to be $ n(a + b\sqrt{2}) = |a^2 - 2b^2| $.

First, we show that we can basically interpret $ a + b\sqrt{2} $ as the following matrix $\left(\begin{array}{cc}a & b\sqrt{2} \\ b\sqrt{2} & a \end{array}\right) $, Indeed, addition of values correspond to addition of matrix, and multiplication of values correspond to multiplication of matrix.

Now since $ a^2 - 2b^2 $ is simply the determinant of the matrix, therefore we know the norm $ |a^2 - 2b^2| $ is multiplicative. We applied the absolute value just to make sure it is non-negative as required by the Euclidean function.

To perform the division, we allow $ e, f \in \mathbf{Q} $ and then we perform some rounding:

\begin{eqnarray*}
  \frac{a+b\sqrt{2}}{c + d\sqrt{2}} &=& e + f\sqrt{2} \\
                                    &=& [e] + [f]\sqrt{2} + (e - [e]) + (f - [f])\sqrt{2} \\
  a+b\sqrt{2} &=& (c + d\sqrt{2})([e] + [f]\sqrt{2}) + (c + d\sqrt{2})((e - [e]) + (f - [f])\sqrt{2}) \\
\end{eqnarray*}

Therefore, we would like to bound the norm of the remainder $ (c + d\sqrt{2})((e - [e]) + (f - [f])\sqrt{2}) $. But since the norm is multiplicative, it suffice to show the norm of $ ((e - [e]) + (f - [f])\sqrt{2}) $ is less than 1. Indeed, the norm is

\begin{eqnarray*}
  & & ((e - [e]) + (f - [f])\sqrt{2}) \\
  &=& |(e - [e])^2 - 2(f - [f])^2| \\
  &\le& \frac{1}{2}
\end{eqnarray*}

The last inequality comes from the fact that $ |x - [x]| \le \frac{1}{2} $ and therefore the maximum is attained when $ e = [e] $ and $ |f - [f]| = \frac{1}{2} $.

Therefore $ \mathbf{Z}[\sqrt{2}] $ is Euclidean.

The units of $ \mathbf{Z}[\sqrt{2}] $ is simply the values with $ n(u) = 1 $. This is because any unit must have $ n(u) = 1 $ because the norm is multiplicative. $ n(u) > 1 $ and $ n(u)n(u^{-1}) = 1 $ would then dictate $ n(u^{-1}) $ to be a fraction, which is impossible. On the other hand, if $ n(u) = 1 $, then $ (a + b\sqrt{2})(a - b\sqrt{2}) = a^2 - 2b = \pm 1 $, and therefore we can easily find the inverse of $ u $. 

With that, the units are given by the Pell's equation, which is $ \pm(1+\sqrt{2})^n $.

Since $ \mathbb{Z}[\sqrt{2}] $ is Euclidean, it is also a unique factorization domain. To find the primes in , we consider the prime factorization of the norm. Here we can use the theory of generalized Pell's equation. In particular, this lemma is useful.

For prime $ p $, $ x^2 - 2y^2 = p $ has solutions if and only if $ 2 $ is a quadratic residue mod $ p $.

Now suppose the norm of $ v $ is a prime $ p $, in this case, $ v $ cannot be a product of more than one non unit element. So $ n(v) = n(a + b\sqrt{2}) = a^2 - 2b^2 = p $, and therefore $ 2 $ must be a quadratic residue mod $ p $.

Now suppose the norm of $ v $ is $ p^2 $ where $ p $ is a prime and 2 is not a quadratic residue mod $ p $. Now if $ v = mn $ can be factorized into non units, then $ m\overline{m}n\overline{n} = p^2 $ has 4 non unit factors on the left hand side but only two prime factors on the right hand side, that contradicts unique factorization. Therefore $ v $ is a prime and cannot be factorized. In that case, $ v $ has to be the integer $ p $.

Otherwise, the value cannot be prime. If we consider the prime factorization on both sides, the left hand side of $ v\overline{v} $ has only two factors with identical norm but the left hand side will have at least two factors with distinct norms, which again contradicts unique factorization.

That shows the primes in $ \mathbf{Z}[\sqrt{2}] $ are either integer primes with 2 not a quadratic residue mod $ p $ or solutions to the generalized Pell's equation $ x^2 - 2y^2 = p $ with $ p $ a quadratic mod $ p $.