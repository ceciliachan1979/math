\documentclass{article}
\usepackage{amssymb}
\usepackage{nth}
\usepackage[utf8]{inputenc}
\usepackage{graphicx}
\usepackage[final]{pdfpages}


\title{Number Theory}
\author{Cecilia Chan}
\date{June 2021}

\begin{document}

\includepdf[pages=-]{Exams/NumberTheory/Problems/Elementary_Number_Theory_Final_Exam__2021_Spring_.pdf}

\maketitle
\section*{Problem 1}

\subsection*{Solution by Cecilia}


Consider $ N = p_1^{e_1} p_2^{e_2} \cdots p_n^{e_n} $ ($ p_1 < p_2 \cdots < p_n $ are prime), we claim that only when $ n = 1 $ (i.e. $ N $ is a prime power) satisfy the given conditions.

\begin{proof}
It is obvious that when $ n = 1 $, $ k = n + 1 $ and $ d_k = p_1^{k - 1} $, and therefore the divisibility constraint is trivially satisifed as follows:

\begin{eqnarray*}
    p_1^{i - 1} = d_i | d_{i + 1} + d_{i + 2} = p_1^i + p_1^{i + 1} = p_1^{i-1}(p + p^2)
\end{eqnarray*}

Suppose $ n \ge 2 $, consider $ s = \frac{N}{p_1^{e_1}} = p_2^{e_2} \cdots p_n^{e_n} $. Notice obviously that $ s \ne d_k = N $ and $ s \ne d_1 = 1 $, so let $ s = d_x $ for some $ 1 < x < k $.

First, consider an arbitrary factor $ E $ of $ N $ that is a multiple of $ p_1 $, it must has the form $ E = p_1^{f_1} p_2^{f_2} \cdots p_n^{f_n} $ where $ 0 < f_1 < e_1 $ and $ 0 \le f_i \le e_i $ for all $ 1 < i \le n $.

Suppose $ d_{x+1} $ is not divisible by $ p_1 $, then $ d_{x+1} = p_2^{f_2} \cdots p_n^{f_n} $, but $ s = p_2^{e_2} \cdots p_n^{e_n} $ already have all the largest exponents, so there is no way $ d_{x+1} > d_x $, that is a contradiction and so $ d_{x+1} $ is divisible by $ p_1 $.

On the other hand, suppose $ d_{x-1} $ is not divisible by $ p_1 $, then $ d_{x-1} = p_2^{f_2} \cdots p_n^{f_n} $. Since $ d_{x-1} < d_x $, so $ f_i \le e_i $ with at least one $ j $ such that $ f_j < e_j $.

Now we find 

\begin{align*}
      p_2^{f_2} \cdots p_n^{f_n} &<   p_1 p_2^{f_2} \cdots p_n^{f_n} \\
                                 &<   \frac{p_j}{p_1} p_1 p_2^{f_2} \cdots p_n^{f_n} &\text{It is legal to multiply $ p_j $ because $ f_j < e_j $} \\
                                 &=   p_j p_2^{f_2} \cdots p_n^{f_n} \\
                                 &\le s &\text{Because $ s $ has the biggest exponents for all $ p_2 \cdots p_n $} \\
                                 &=   d_x
\end{align*}

And that's a contradiction because there is a factor between $ d_{x-1} $ and $ d_x $

Now we have shown that both $ d_{x - 1} $ and $ d_{x + 1} $ are divisible by $ p_1 $, but obviously $ d_x $ doesn't, so $ d_{x-1} | d_x + d_{x+1} $ is a contradiction, and we have proved $ n = 1 $.

\end{proof}

\subsection*{Solution by aoiyamada211, written up by CY Fung}

\begin{proof}

Firstly, note that (1) $d_i d_{k+1-i} = N$; (2) $d_2$ must be the smallest prime factor of $N$. Let $p_a = d_2$.

\hspace{3em}

Consider the nature of $d_3$. An integer larger than 1 is either prime or composite.

\hspace{3em}

Case I: $d_3$ is prime.

Let $d_3 = p_b$. By the condition $d_{i+1} > d_i$, we have $p_b > p_a$. Consider $d_k = N$, $d_{k-1} = \frac{N}{p_a}$ and $d_{k-2} = \frac{N}{p_b}$; by the condition $d_i | d_{i+1} + d_{i+2}$, we have $ m \frac{N}{p_b} = N(1+\frac{1}{p_a}) $, where $m$ is an integer. Then $ m p_a = p_b (1 + p_a)$, since $1+p_a$ and $p_a$ is coprime, $p_a | p_b$, but this contradicts $p_a$ and $p_b$ are distinct and prime. !!!

\hspace{3em}

Case II: $d_3$ is composite.

A composite number must be the product of two or more primes (not necessarily distinct). Let $d_3 = m p_b p_c $, we have $d_3 > p_b > 1 = d_1 $, $d_3 > p_c > 1 = d_1$; but there are no factors of $N$ in between $d_2 = p_a$ and $d_3$, in addition, $p_b | N$, $p_c | N$, so $p_b = d_g$ for an integer $g$ satisfying $1 < g \le 3$ and $p_c = d_h$ for an integer $h$ satisfying $1 < h \le 3$. But $p_b \ne d_3$, $p_c \ne d_3$, leads to $g = h = 2$, so $p_a = p_b = p_c$ and $d_3 = p_a^2$.

\hspace{3em}

Since Case I cannot happen and Case II is valid, Case II is our only choice.

\hspace{3em}

By the condition of the question, $k > 2$. For case $k = 3$, we have $N$ is the square of a prime. Consider $k > 3$.

\hspace{3em}

Let $P(t)$ be the statement ``for an integer $N$ satisfying the required coniditions, $d_t = d_2^{t-1}$''. 

\hspace{3em}

Assume $P(r)$ is true for all positive integer $r \le i$. Let $p = d_2$. $d_{i-1} = p^{i-2}$ and $d_i = p^{i-1}$. $d_{i-1} | d_i + d_{i+1}$. Since $p^{i-2} | d_i$, $p^{i-2} | d_{i+1}$. Let the integer $m = \frac{d_{i+1}}{p^{i-2}}$. Since $m$ is a (proper) factor of $d_{i+1}$, it must also be a factor of $N$, therefore $m = d_h$ for some integer $1 \le h < i+1$. But we have assumed that $d_r = p^{r-1}$ for $1 \le r \le i$, so $d_{i+1} = p^{i-2+x}$ for some positive integer $x > 0$. If $x > 2$, there would be some factor(s) of $N$ between $d_i$ and $d_{i+1}$, contradicting the given condition. And again, $d_i$ and $d_{i+1}$ are distinct, so $x \ne 1$, then we got $x = 2$. Then we have $d_{i+1} = p^i$ and $P(r+1)$ is also true.

\hspace{3em}

From the previous arguments and the given condition $d_1 = 1 = p^0$, we know $P(1)$, $P(2)$ and $P(3)$ are true. 

\hspace{3em}

From the conclusion of the induction we have $ N = d_k = p^{k-1} $.

\hspace{3em}

Summing up, integers satisfying the given conditions are powers of primes.

\end{proof}


\subsection*{(2)}
\begin{eqnarray*}
         6x^8 & \equiv & 5   \pmod{13} \\
  2^5 (2^y)^8 & \equiv & 2^9 \pmod{13} \\
       5 + 8y & \equiv & 9   \pmod{12} \\
           8y & \equiv & 4   \pmod{12} \\
            y & \equiv & 2   \pmod{12} \\
            x & \equiv & 4   \pmod{13}
\end{eqnarray*}
\begin{verbatim}
>>> (6 * 4 ** 8) % 13
5
\end{verbatim}

\subsection*{(3)}
We observe that $ x \equiv 1 \pmod{13} $ is an obvious answer. By long division (in modular arithmetic of course) we get $ 2x^2 + 6x + 5 = (x - 1)(2x + 8) $. That gives the another solution to be $ x \equiv 9 \pmod{13} $.
\begin{verbatim}
>>> (2 * (1 ** 2) + 6 * 1 + 5) % 13
0
>>> (2 * (9 ** 2) + 6 * 9 + 5) % 13
0
\end{verbatim}

\section*{Question 2}
\subsection*{(1)}
We can represent the number $ 1 \le m < p $ as elements of $ Z_p $. Since $ Z_p $ is cyclic, we can represent them as distinct power of a primitive root $ r $.

\begin{eqnarray*}
  &      & S                                         \\
  &\equiv& \sum\limits_{i = 1}^{p-1}m^n     \pmod{p} \\
  &\equiv& \sum\limits_{i = 1}^{p-1}(r^i)^n \pmod{p} \\
  &\equiv& \sum\limits_{i = 1}^{p-1}(r^n)^i
\end{eqnarray*}

When $ p - 1 | n $, we have $ r^n \equiv 1 \pmod{p} $ by Fermat's theorem. Therefore $ S \equiv p - 1 \equiv -1 \pmod{p} $. Otherwise

\begin{eqnarray*}
  & & S                                              \\
  &\equiv& \sum\limits_{i = 1}^{p-1}(r^n)^i \pmod{p} \\
  &\equiv& \frac{(r^n)^p - r^n}{r^n - 1}    \pmod{p} \\
  &\equiv& \frac{r^n - r^n}{r^n - 1}        \pmod{p} \\
  &\equiv& 0                                \pmod{p} 
\end{eqnarray*}
The \nth{3} equivalence is a consequence of Fermat's theorem.
\subsection*{(2)}
\begin{eqnarray*}
  &      & \sum\limits_{m = 1}^{p-1}f(m) \\
  &\equiv& \sum\limits_{m = 1}^{p-1}\sum\limits_{i = 0}^{k}a_{k-i} m^i \pmod{p} \\
  &\equiv& \sum\limits_{i = 0}^{k}\sum\limits_{m = 1}^{p-1}a_{k-i} m^i \pmod{p} \\
  &\equiv& \sum\limits_{i = 0}^{k}a_{k-i}\sum\limits_{m = 1}^{p-1} m^i \pmod{p} \\
  &\equiv& \sum\limits_{i = 0}^{k}a_{k-i} 0 \pmod{p} \\
  &\equiv& 0
\end{eqnarray*}
The \nth{4} equivalence is a consequence of part 1 as $ k < p - 1 $.

\section*{Question 3}
Manually computing the Legendre's symbol is simply too tedious, the following program is written to leverage quadratic reciprocity to compute Legendre's symbol and have it spill out the \LaTeX code.
\begin{verbatim}
def prime_legendre(a, p):
  if a == 2:
    p8 = p % 8
    if p8 == 1 or p8 == 7:
      answer = 1
    else:
      answer = -1
    print("\\begin{eqnarray*}")
    print("  & & \\left(\\frac{%s}{%s}\\right) \\\\" % (a, p))
    print("  &=& (-1)^{(%s^2 - 1)/8} \\\\" % p)
    print("  &=& %s" % answer)
    print("\\end{eqnarray*}")
    return answer
  else:
    answer = legendre(p % a, a) * (-1)** ((p-1)//2 * (a-1)//2)
    print("\\begin{eqnarray*}")
    print("  & &  \\left(\\frac{%s}{%s}\\right) \\\\" % (a, p))
    print("  &=& \\left(\\frac{%s}{%s}\\right) (-1)^{(\\frac{%s-1}{2}\\frac{%s-1}{2})} \\\\" % (p, a, p, a))
    print("  &=& \\left(\\frac{%s}{%s}\\right) (-1)^{(\\frac{%s-1}{2}\\frac{%s-1}{2})} \\\\" % (p % a, a, p, a))
    print("  &=& %s" % answer)
    print("\\end{eqnarray*}")
    return answer

def legendre(a, p):
  if a == 1:
    return 1
  else:
    s = "  & & \\left(\\frac{%s}{%s}\\right) \\\\\n  &=& " % (a, p)
    terms = []
    answer = 1
    factor = 2
    while a > 1:
      while factor <= a:
        power = 0
        while a % factor == 0:
          a = a // factor
          power = power + 1        
        if power > 0:
          terms.append((factor, power))
          answer = answer * (prime_legendre(factor, p) ** power)
          factor = factor + 1
          break
        else:
          factor = factor + 1
  if len(terms) > 1:
    for (f, power) in terms:
      if power == 1:
        s = s + "\\left(\\frac{%s}{%s}\\right)" % (f, p)
      else:
        s = s + "\\left(\\frac{%s}{%s}\\right)^%s" % (f, p, power)
    s = s + " \\\\\n  &=& %s" % answer
    print("\\begin{eqnarray*}")
    print(s)
    print("\\end{eqnarray*}")
  return answer    
\end{verbatim}
\subsection*{(1)}
\begin{eqnarray*}
  & & \left(\frac{2}{97}\right) \\
  &=& (-1)^{(97^2 - 1)/8} \\
  &=& 1
\end{eqnarray*}
\begin{eqnarray*}
  & & \left(\frac{2}{19}\right) \\
  &=& (-1)^{(19^2 - 1)/8} \\
  &=& -1
\end{eqnarray*}
\begin{eqnarray*}
  & &  \left(\frac{19}{97}\right) \\
  &=& \left(\frac{97}{19}\right) (-1)^{(\frac{97-1}{2}\frac{19-1}{2})} \\
  &=& \left(\frac{2}{19}\right) (-1)^{(\frac{97-1}{2}\frac{19-1}{2})} \\
  &=& -1
\end{eqnarray*}
\begin{eqnarray*}
  & & \left(\frac{38}{97}\right) \\
  &=& \left(\frac{2}{97}\right)\left(\frac{19}{97}\right) \\
  &=& -1
\end{eqnarray*}
Therefore $ x^2 \equiv 38 \pmod{97} $ does not have a solution.
\subsection*{(2)}
\begin{eqnarray*}
  & &  \left(\frac{3}{157}\right) \\
  &=& \left(\frac{157}{3}\right) (-1)^{(\frac{157-1}{2}\frac{3-1}{2})} \\
  &=& \left(\frac{1}{3}\right) (-1)^{(\frac{157-1}{2}\frac{3-1}{2})} \\
  &=& 1
\end{eqnarray*}
\begin{eqnarray*}
  & & \left(\frac{2}{3}\right) \\
  &=& (-1)^{(3^2 - 1)/8} \\
  &=& -1
\end{eqnarray*}
\begin{eqnarray*}
  & &  \left(\frac{3}{11}\right) \\
  &=& \left(\frac{11}{3}\right) (-1)^{(\frac{11-1}{2}\frac{3-1}{2})} \\
  &=& \left(\frac{2}{3}\right) (-1)^{(\frac{11-1}{2}\frac{3-1}{2})} \\
  &=& 1
\end{eqnarray*}
\begin{eqnarray*}
  & &  \left(\frac{11}{157}\right) \\
  &=& \left(\frac{157}{11}\right) (-1)^{(\frac{157-1}{2}\frac{11-1}{2})} \\
  &=& \left(\frac{3}{11}\right) (-1)^{(\frac{157-1}{2}\frac{11-1}{2})} \\
  &=& 1
\end{eqnarray*}
\begin{eqnarray*}
  & & \left(\frac{33}{157}\right) \\
  &=& \left(\frac{3}{157}\right)\left(\frac{11}{157}\right) \\
  &=& 1
\end{eqnarray*}
Therefore $ (x + 2) \equiv 33 \pmod{157} $ has two solutions, and so does $ x^2 + 4x \equiv 29 \pmod{157} $
\begin{verbatim}
>>> (68 * 68 + 4 * 68) % 157
29
>>> (85 * 85 + 4 * 85) % 157
29
\end{verbatim}
\subsection*{(3)}
\begin{eqnarray*}
  & & \left(\frac{2}{383}\right) \\
  &=& (-1)^{(383^2 - 1)/8} \\
  &=& 1
\end{eqnarray*}
\begin{eqnarray*}
  & & \left(\frac{2}{5}\right) \\
  &=& (-1)^{(5^2 - 1)/8} \\
  &=& -1
\end{eqnarray*}
\begin{eqnarray*}
  & &  \left(\frac{5}{7}\right) \\
  &=& \left(\frac{7}{5}\right) (-1)^{(\frac{7-1}{2}\frac{5-1}{2})} \\
  &=& \left(\frac{2}{5}\right) (-1)^{(\frac{7-1}{2}\frac{5-1}{2})} \\
  &=& -1
\end{eqnarray*}
\begin{eqnarray*}
  & &  \left(\frac{7}{383}\right) \\
  &=& \left(\frac{383}{7}\right) (-1)^{(\frac{383-1}{2}\frac{7-1}{2})} \\
  &=& \left(\frac{5}{7}\right) (-1)^{(\frac{383-1}{2}\frac{7-1}{2})} \\
  &=& 1
\end{eqnarray*}
\begin{eqnarray*}
  & & \left(\frac{2}{3}\right) \\
  &=& (-1)^{(3^2 - 1)/8} \\
  &=& -1
\end{eqnarray*}
\begin{eqnarray*}
  & &  \left(\frac{3}{17}\right) \\
  &=& \left(\frac{17}{3}\right) (-1)^{(\frac{17-1}{2}\frac{3-1}{2})} \\
  &=& \left(\frac{2}{3}\right) (-1)^{(\frac{17-1}{2}\frac{3-1}{2})} \\
  &=& -1
\end{eqnarray*}
\begin{eqnarray*}
  & &  \left(\frac{17}{383}\right) \\
  &=& \left(\frac{383}{17}\right) (-1)^{(\frac{383-1}{2}\frac{17-1}{2})} \\
  &=& \left(\frac{9}{17}\right) (-1)^{(\frac{383-1}{2}\frac{17-1}{2})} \\
  &=& 1
\end{eqnarray*}
\begin{eqnarray*}
  & & \left(\frac{238}{383}\right) \\
  &=& \left(\frac{2}{383}\right)\left(\frac{7}{383}\right)\left(\frac{17}{383}\right) \\
  &=& 1
\end{eqnarray*}
Therefore $ (x + 380)^2 \equiv 238 \pmod{383} $ has two solutions, and so does $ x^2 + 377x + 154 \equiv 0 \pmod{383} $. The transformation from $ x^2 + 377x + 154 \equiv 0 \pmod{383} $ to $ (x + 380)^2 \equiv 238 \pmod{383} $ is done simply by completing the square. Notice $ 2^{-1} \pmod{383} = 192 $ so we find $ 380 = 192 \times 377 \pmod{383} $. The rest is obvious.

\begin{verbatim}
>>> (94 ** 2 + 377 * 94 + 154) % 383
0
>>> (295 ** 2 + 377 * 295 + 154) % 383
0
\end{verbatim}

\section*{Question 4}
\subsection*{(1)}
\begin{eqnarray*}
  & & \left(\frac{-6}{p}\right) \\
  &=& \left(\frac{-1}{p}\right)\left(\frac{2}{p}\right)\left(\frac{3}{p}\right) \\
  &=& \left(\frac{-1}{p}\right)\left(\frac{2}{p}\right)\left(\frac{p}{3}\right)(-1)^{\frac{3-1}{2}\frac{p-1}{2}} \\
  &=& A \times B \times C \times D
\end{eqnarray*}
Now we evaluate the values:
\begin{eqnarray*}
  \begin{array}{cccccc}
    p \pmod{24}  &  A &  B &  C &  D & \left(\frac{-6}{p}\right) \\
               1 &  1 &  1 &  1 &  1 &  1 \\
               5 &  1 & -1 & -1 &  1 &  1 \\
               7 & -1 &  1 &  1 & -1 &  1 \\
              11 & -1 & -1 & -1 & -1 &  1 \\
              13 &  1 & -1 &  1 &  1 & -1 \\
              17 &  1 &  1 & -1 &  1 & -1 \\
              19 & -1 & -1 &  1 & -1 & -1 \\
              23 & -1 &  1 & -1 & -1 & -1 \\
  \end{array}
\end{eqnarray*}
So now we verify $ \left(\frac{-6}{p}\right) = 1 \iff p \pmod{12} \in \{1, 5, 7, 11\}$.
\subsection*{(2)}
Suppose there are only finitely many primes of form $ 24k + 5 $ or $ 24k + 11 $. Let $ \{p_1, p_2, \cdots p_n \} $ be all of them and consider $ q $ to be a prime factor of $ N = 9(p_1 p_2 \cdots p_n)^2 + 6 $. 

\begin{eqnarray*}
  (3(p_1 p_2 \cdots p_n))^2 + 6 &=& N \\
  (3(p_1 p_2 \cdots p_n))^2 &=& N - 6 \\
  (3(p_1 p_2 \cdots p_n))^2 &=& - 6 \pmod{N} \\
  (3(p_1 p_2 \cdots p_n))^2 &=& - 6 \pmod{q} \\
  \left(\frac{-6}{q}\right) &=& 1 \\
  q &=& \{1, 5, 7, 11\} \pmod{24}
\end{eqnarray*}

Note that the product of any set of numbers that is either $ 1 \pmod{24} $ or $ 7 \pmod{24} $ is either $ 1 \pmod{24} $ or $ 7 \pmod{24} $, but $ N = 15 \pmod{24} $. Therefore, there must be a prime factor $ r $ of $ N $ which is either $ 5 \pmod{24} $ or $ 11 \pmod{24} $. Note that $ r $ is a factor of $ N $ and therefore cannot be one of $ \{ p_1, p_2, \cdots p_n \} $. This contradicts the fact that $ \{p_1, p_2, \cdots p_n \} $ are all the primes of form $ 24k + 5 $ or $ 24k + 11 $. Therefore, there must be infinitely many primes of form $ 24k + 5 $ or $ 24k + 11 $.

\section*{Question 5}
Since $ x^2 + y^2 = (z^2)^2 $ is a primitive pythagorean triple, we can let $ m > n, (m, n) = 1 $ such that $ x = m^2 - n^2, y = 2mn, z^2 = m^2 + n^2 $. Now we can use the same trick again and let $ s > t, (s, t) = 1 $ such that $ m = s^2 - t^2, n = 2st, z = s^2 + t^2 $. Now we get $ x = m^2 - n^2 = (s^2 - t^2)^2 - (2st)^2 = s^4 - 6s^2t^2 + t^4 $ and $ y = 2mn = 2(s^2 - t^2)2st = 4s^3t - 4st^3 $.

\section*{Question 6}
The repetitive work is best left for the computer, here is the source code for computing finite continued fraction.
\begin{verbatim}
def finite_continued_fraction(n, d):
  result = []
  finite_continued_fraction_helper(n, d, result)
  return result

def finite_continued_fraction_helper(n, d, l):
  if d == 0:
    return
  i = n // d
  l.append(i)
  finite_continued_fraction_helper(d, n - i * d, l)

def expression(continued_fraction_list):
  if len(continued_fraction_list) == 1:
    return str(continued_fraction_list[0])
  else:
    head = continued_fraction_list[0]
    tail = list(continued_fraction_list)
    tail.pop(0)
    return "%s + 1/(%s)" % (head, expression(tail))

def value(continued_fraction_list):
    if len(continued_fraction_list) == 1:
        return (continued_fraction_list[0], 1)
    else:
        head = continued_fraction_list[0]
        tail = list(continued_fraction_list)
        tail.pop(0)
        (n, d) = value(tail)
        return (head * n + d, n)
\end{verbatim}

And here are the results:
\begin{verbatim}
>>> finite_continued_fraction(53, 75)
[0, 1, 2, 2, 2, 4]

>>> finite_continued_fraction(117, 41)
[2, 1, 5, 1, 5]

>>> value([2, 3, 1, 4, 3, 5])
(733, 324)

>>> value([1, 3, 5, 7, 9, 2])
(2911, 2217)
\end{verbatim}

\section*{Question 7}
Again, we will leave the repetitive work to the computer, here is the source code for computing continued fraction and their convergents for an quadratic irrational $ \frac{p + \sqrt{r}}{q} $.
\begin{verbatim}
import math

def quadratic_continued_fraction(p, d, q):
    p *= q
    q *= q
    d *= q
    solution = []
    memo = {}
    start = -1
    while True:
        if (p, q) in memo:
            start = memo[(p, q)]
            break
        alpha = (p + math.sqrt(d))/q
        a = int(alpha)
        memo[(p, q)] = len(solution)
        solution.append(a)
        np = a * q - p
        nq = (d - np * np)//q
        p = np
        q = nq
    return (solution, start)

def convergents(continued_fraction_list):
    p0 = continued_fraction_list[0]
    q0 = 1
    yield (p0, q0)
    q1 = continued_fraction_list[1]
    p1 = p0 * q1 + 1
    yield (p1, q1)
    for i in range(2, len(continued_fraction_list)):
      a = continued_fraction_list[i]
      p2 = a * p1 + p0
      q2 = a * q1 + q0
      yield (p2, q2)
      (p0, p1, q0, q1) = (p1, p2, q1, q2)
\end{verbatim}

And here is the result:
\begin{verbatim}
>>> quadratic_continued_fraction(0, 34, 1)
([5, 1, 4, 1, 10], 1)

>>> for convergent in convergents([5, 1, 4, 1, 10]):
...     print(convergent)
...
(5, 1)
(6, 1)
(29, 5)
(35, 6)
(379, 65)
>>>
\end{verbatim}

That implies the continued fraction representation for $ \sqrt{34} = [5;\overline{1,4,1,10}] $, and the fundamental solution of $ x^2 - 34y^2 = 1 $ is given by $ x = 35 $ and $ y = 6. $. We pick that convergent because the period is 4.

\end{document}
