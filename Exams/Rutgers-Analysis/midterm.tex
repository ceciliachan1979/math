\documentclass{article}
\usepackage{amssymb}
\usepackage{nth}
\usepackage[utf8]{inputenc}
\usepackage{graphicx}
\usepackage[final]{pdfpages}


\title{Number Theory}
\author{Cecilia Chan}
\date{February 2024}

\begin{document}
\section*{Problem 1}
\subsection*{Part 1}
True, because otherwise the real number would be the disjoint union of two countable sets.
\subsection*{Part 2}
Yes, because it is the union of countably many countable sets.
\subsection*{Part 3}
It depends on the space. If the space is complete (e.g. $ \mathbb{R}^N $), then the sequence is convergent (that's the definition of a space being complete).
\subsection*{Part 4}
False, consider the sequence $ \{1, -1, 1, \ldots \} $, it is bounded but not convergent.
\subsection*{Part 5}
True, to satisfy the epsilon challenge, just pick the maximum $ N $ for both subsequences.
\subsection*{Part 6}
True, because if that's not true, $ t - \epsilon $ is a smaller upper bound.
\subsection*{Part 7}
True, to satisfy the epsilon challenge, just pick the $ N $ corresponding to $ \frac{1}{L} \epsilon $ for $ f(x) $ where $ -L < g(x) < L$.
\subsection*{Part 8}
True, because otherwise at least one of the greatest lower bounds is actually not greatest.
\subsection*{Part 9}
True, the least upper bound exists and it is the limit.
\subsection*{Part 10}
False, 0 is also an accumulation point.

\section*{Problem 2}
Let $ N = \lceil \frac{2}{\epsilon} \rceil $, then for $ n > N $, we have
\begin{eqnarray*}
              n & > & N \\
              n & > & \frac{2}{\epsilon} \\
       \epsilon & > & \frac{2}{N} \\
    \frac{2}{n} & < & \epsilon \\
    \frac{2}{n} - 3 + 3 & < & \epsilon \\
    \frac{2}{n} - \frac{3n}{n} + 3 & < & \epsilon \\
    -(\frac{3n - 2}{n} - 3) & < & \epsilon
\end{eqnarray*}

We also have $ \frac{3n - 2}{n} = 3 - \frac{2}{n} < 3 $, section

\begin{eqnarray*}
      \frac{3n - 2}{n} &=& 3 - \frac{2}{n} < 3 \\
  \frac{3n - 2}{3} - 3 &<& 0 \\
                       &<& \epsilon
\end{eqnarray*}

Together with have $ |\frac{3n - 2}{3} - 3| < \epsilon $ when $ n > \lceil \frac{2}{\epsilon} \rceil $, so we conclude

\begin{eqnarray}
    \lim_{n \to \infty} \frac{3n - 2}{n} = 3
\end{eqnarray}

\section*{Problem 3}
Let $ \delta = \min (1, \frac{\epsilon}{8}) $, note that $ 0 < \delta < 1 $, so $ 0 < \delta^2 < \delta $.

Whenever $ | x - 1 | < \delta $, we have

\begin{eqnarray*}
    -\delta &< x - 1 &< \delta \\
    2 -\delta &< x + 1 &< 2 + \delta \\
    (2 -\delta)^2 &< (x + 1)^2 &< (2 + \delta)^2 \\
    4 - 4\delta + \delta^2 &< x^2 + 2x + 1 &< 4 + 4\delta + \delta^2 \\
    -4\delta + \delta^2 &< (x^2 + 2x + 1) - 4 &< 4\delta + \delta^2 \\
    -4\delta &< (x^2 + 2x + 1) - 4 &< 5\delta \\
    -\frac{\epsilon}{2} &< (x^2 + 2x + 1) - 4 &< \frac{5\epsilon}{8} \\
    -\epsilon &< (x^2 + 2x + 1) - 4 &< \epsilon
\end{eqnarray*}

So we conclude that $ \lim_{x \to 1} x^2 + 2x + 1 = 4 $.

\section*{Problem 4}

We claim that the limit is $ L = \frac{1 + \sqrt{17}}{2} $

Note that when $ L $ is designed to be the larger root of $ x^2 - x - 4 $, so when $ x > L $, $ x^2 - x - 4 > 0 $, and so we have:

\begin{eqnarray*}
x^2 - x - 4 > 0
x^2 > x + 4
x > \sqrt{4 + x}
\end{eqnarray*}

Since $ a_1 = 4 > L $, the iteration will be decreasing. It is obviously bounded below by 0, so a limit exists.

Suppose the limit exists, then 

\begin{eqnarray*}
      lim_{n \to \infty} a_n \\
  &=& lim_{n \to \infty} \sqrt{4 + a_{n-1}} \\
  &=& \sqrt{4 + lim_{n \to \infty} a_{n-1}} \\
\end{eqnarray*}

So it must be a root of $ x = \sqrt{4 + x} $, and apparently it must be > 0, so it must be $ L $.

\section*{Problem 5}
\subsection*{Part a}
The use of L'Hopital's rule is justified by the $ \frac{\infty}{\infty} $ form of the limit. We have:
\begin{eqnarray*}
  & & \lim_{n \to \infty}\frac{3n^3 - n + 100}{n^3 - n + 2} \\
  &=& \lim_{n \to \infty}\frac{9n^2 - 1}{3n^2 - 1} \\
  &=& \lim_{n \to \infty}\frac{18n}{6n} \\
  &=& 3 
\end{eqnarray*}

\subsection*{Part b}
The use of L'Hopital's rule is justified by the $ \frac{0}{0} $ form of the limit. We have:
\begin{eqnarray*}
  & & \lim_{x \to 0}\frac{\sqrt{x + 4} - 2}{x} \\
  &=& \lim_{x \to 0}\frac{\frac{1}{2}(x + 4)^{-\frac{1}{2}}}{1} \\
  &=& \frac{1}{4} 
\end{eqnarray*}

\section*{Problem 6}
$ \lim_{x_0 \to 0} f(x) = 2 $ implies for any $ \epsilon > 0 $, there exists $ \delta > 0 $ such that $ 0 < |x - x_0| < \delta $ implies $ |f(x) - 2| < \epsilon $. In particular, for $ \epsilon = 1 $, there exists $ \delta > 0 $ such that $ 0 < |x - x_0| < \delta $ implies $ |f(x) - 2| < 1 $.

Note that $ 0 < | x - x_0| < \delta $ is the same as $ x \in (x_0 - \delta, x_0 + \delta) $, and $ |f(x) - 2| < 1 $ is the same as $ f(x) \in (1, 3) $, in particular, $ f(x) > 1 $.

The fact that $ x_0 $ is an accumulation point implies some $ x $ will exist in the given set. It is not strictly necessary because otherwise the statement would simply be vacuously true.

\section*{Problem 7}
The Bolzano-Weierstrass theorem states that every bounded sequence has a convergent subsequence. The sequence $ \{a_n\} $ is bounded, so it has a convergent subsequence $ \{a_{n_k}\} $.

We can prove that by showing a monotone subsequence exists, because then the sequence of monotone and bounded, so it must converge.

Assuming the sequence has a monotonic increasing suffix, then we can take that suffix.

Otherwise it must have a maximum, so we can take that point. The rest of the sequence is not monotonically increasing either, so we can build the sequence inductively.

By definition, this sequence is monotonically decreasing. 

So we know there exists a monotonic subsequence, and it is also bounded, so it must converge.

\end{document}
