\section*{Problem 1}

\subsection*{Solution by Cecilia}


Consider $ N = p_1^{e_1} p_2^{e_2} \cdots p_n^{e_n} $ ($ p_1 < p_2 \cdots < p_n $ are prime), we claim that only when $ n = 1 $ (i.e. $ N $ is a prime power) satisfy the given conditions.

\begin{proof}
It is obvious that when $ n = 1 $, $ k = n + 1 $ and $ d_k = p_1^{k - 1} $, and therefore the divisibility constraint is trivially satisifed as follows:

\begin{eqnarray*}
    p_1^{i - 1} = d_i | d_{i + 1} + d_{i + 2} = p_1^i + p_1^{i + 1} = p_1^{i-1}(p + p^2)
\end{eqnarray*}

Suppose $ n \ge 2 $, consider $ s = \frac{N}{p_1^{e_1}} = p_2^{e_2} \cdots p_n^{e_n} $. Notice obviously that $ s \ne d_k = N $ and $ s \ne d_1 = 1 $, so let $ s = d_x $ for some $ 1 < x < k $.

First, consider an arbitrary factor $ E $ of $ N $ that is a multiple of $ p_1 $, it must has the form $ E = p_1^{f_1} p_2^{f_2} \cdots p_n^{f_n} $ where $ 0 < f_1 < e_1 $ and $ 0 \le f_i \le e_i $ for all $ 1 < i \le n $.

Suppose $ d_{x+1} $ is not divisible by $ p_1 $, then $ d_{x+1} = p_2^{f_2} \cdots p_n^{f_n} $, but $ s = p_2^{e_2} \cdots p_n^{e_n} $ already have all the largest exponents, so there is no way $ d_{x+1} > d_x $, that is a contradiction and so $ d_{x+1} $ is divisible by $ p_1 $.

On the other hand, suppose $ d_{x-1} $ is not divisible by $ p_1 $, then $ d_{x-1} = p_2^{f_2} \cdots p_n^{f_n} $. Since $ d_{x-1} < d_x $, so $ f_i \le e_i $ with at least one $ j $ such that $ f_j < e_j $.

Now we find 

\begin{align*}
      p_2^{f_2} \cdots p_n^{f_n} &<   p_1 p_2^{f_2} \cdots p_n^{f_n} \\
                                 &<   \frac{p_j}{p_1} p_1 p_2^{f_2} \cdots p_n^{f_n} &\text{It is legal to multiply $ p_j $ because $ f_j < e_j $} \\
                                 &=   p_j p_2^{f_2} \cdots p_n^{f_n} \\
                                 &\le s &\text{Because $ s $ has the biggest exponents for all $ p_2 \cdots p_n $} \\
                                 &=   d_x
\end{align*}

And that's a contradiction because there is a factor between $ d_{x-1} $ and $ d_x $

Now we have shown that both $ d_{x - 1} $ and $ d_{x + 1} $ are divisible by $ p_1 $, but obviously $ d_x $ doesn't, so $ d_{x-1} | d_x + d_{x+1} $ is a contradiction, and we have proved $ n = 1 $.

\end{proof}

\subsection*{Solution by aoiyamada211, written up by CY Fung}

\begin{proof}

Firstly, note that (1) $d_i d_{k+1-i} = N$; (2) $d_2$ must be the smallest prime factor of $N$. Let $p_a = d_2$.

\hspace{3em}

Consider the nature of $d_3$. An integer larger than 1 is either prime or composite.

\hspace{3em}

Case I: $d_3$ is prime.

Let $d_3 = p_b$. By the condition $d_{i+1} > d_i$, we have $p_b > p_a$. Consider $d_k = N$, $d_{k-1} = \frac{N}{p_a}$ and $d_{k-2} = \frac{N}{p_b}$; by the condition $d_i | d_{i+1} + d_{i+2}$, we have $ m \frac{N}{p_b} = N(1+\frac{1}{p_a}) $, where $m$ is an integer. Then $ m p_a = p_b (1 + p_a)$, since $1+p_a$ and $p_a$ is coprime, $p_a | p_b$, but this contradicts $p_a$ and $p_b$ are distinct and prime. !!!

\hspace{3em}

Case II: $d_3$ is composite.

A composite number must be the product of two or more primes (not necessarily distinct). Let $d_3 = m p_b p_c $, we have $d_3 > p_b > 1 = d_1 $, $d_3 > p_c > 1 = d_1$; but there are no factors of $N$ in between $d_2 = p_a$ and $d_3$, in addition, $p_b | N$, $p_c | N$, so $p_b = d_g$ for an integer $g$ satisfying $1 < g \le 3$ and $p_c = d_h$ for an integer $h$ satisfying $1 < h \le 3$. But $p_b \ne d_3$, $p_c \ne d_3$, leads to $g = h = 2$, so $p_a = p_b = p_c$ and $d_3 = p_a^2$.

\hspace{3em}

Since Case I cannot happen and Case II is valid, Case II is our only choice.

\hspace{3em}

By the condition of the question, $k > 2$. For case $k = 3$, we have $N$ is the square of a prime. Consider $k > 3$.

\hspace{3em}

Let $P(t)$ be the statement ``for an integer $N$ satisfying the required coniditions, $d_t = d_2^{t-1}$''. 

\hspace{3em}

Assume $P(r)$ is true for all positive integer $r \le i$. Let $p = d_2$. $d_{i-1} = p^{i-2}$ and $d_i = p^{i-1}$. $d_{i-1} | d_i + d_{i+1}$. Since $p^{i-2} | d_i$, $p^{i-2} | d_{i+1}$. Let the integer $m = \frac{d_{i+1}}{p^{i-2}}$. Since $m$ is a (proper) factor of $d_{i+1}$, it must also be a factor of $N$, therefore $m = d_h$ for some integer $1 \le h < i+1$. But we have assumed that $d_r = p^{r-1}$ for $1 \le r \le i$, so $d_{i+1} = p^{i-2+x}$ for some positive integer $x > 0$. If $x > 2$, there would be some factor(s) of $N$ between $d_i$ and $d_{i+1}$, contradicting the given condition. And again, $d_i$ and $d_{i+1}$ are distinct, so $x \ne 1$, then we got $x = 2$. Then we have $d_{i+1} = p^i$ and $P(r+1)$ is also true.

\hspace{3em}

From the previous arguments and the given condition $d_1 = 1 = p^0$, we know $P(1)$, $P(2)$ and $P(3)$ are true. 

\hspace{3em}

From the conclusion of the induction we have $ N = d_k = p^{k-1} $.

\hspace{3em}

Summing up, integers satisfying the given conditions are powers of primes.

\end{proof}
