\section*{Question 1}
\subsection*{Part a}
\begin{eqnarray*}
  \left|\begin{array}{ccc}
  1 & 1 & -\lambda \\
  1 & \lambda & -1 \\
  \lambda & 1 & -1
  \end{array}\right| &=& 0 \\
  (1)(\lambda)(-1) + (1)(-1)(\lambda) + (-\lambda)(1)(1) - (-\lambda)(\lambda)(\lambda) - (1)(1)(-1) - (1)(-1)(1) &=& 0 \\
  \lambda^3 -3\lambda + 2 &=& 0 \\
  (\lambda - 1)^2(\lambda + 2) &=& 0 
\end{eqnarray*}
Therefore $ \lambda = 1 $ or $ \lambda = -2 $
\subsection*{Part b}
When $ \lambda = 1 $, the only equation we have is $ x + y - z = 0 $, so the answer is $ (x, y, (-x-y)) $.

When $ \lambda = -2 $, the equations become

\begin{eqnarray*}
   x + y + 2z &=& 0 \\
   x - 2y - z &=& 0 \\
  -2x + y - z &=& 0
\end{eqnarray*}

Subtract the first equation by the second equation gives $ 3y + 3z = 0 $, so we have $ y = -z $. Then the first equation implies $ x - z + 2z = 0 $, so $ x = -z $ as well. Therefore the solution is $ (-z, -z, z) $