\section*{Question 8}
\subsection*{Part a}
\begin{eqnarray*}
  & & \left|\begin{array}{ccc}
    a & 1 & b \\
    1 & a & b \\
    1 & 1 & ab
  \end{array}\right| \\
  &=& a^3 b + b + b - ab - ab - ab \\
  &=& b(a^3 -3a + 2) \\
  &=& b(a - 1)^2(a + 2)
\end{eqnarray*}

Therefore (*) has a unique solution if and only if $ a \ne 1 $, $ a \ne -2 $ and $ b \ne 0 $.

Using the first two equations, we have:
\begin{eqnarray*}
  (ax + y + bz) - (x + ay + bz) &=& 0 \\
  (a-1)x - (a - 1)y &=& 0 \\
  x &=& y
\end{eqnarray*}

With that, the last two equations simplifies to
\begin{eqnarray*}
  (a + 1)x + bz &=& 1 \\
  2x + abz &=& b \\
\end{eqnarray*}
So we can scale and solve as follow:
\begin{eqnarray*}
  2(a+1)x + 2bz &=& 2 \\
  2(a+1)x + a(a+1)bz &=& (a+1)b \\
  (2b - a(a+1)b)z &=& 2 - (a+1)b \\
  z &=& \frac{2 - (a+1)b}{2b - a(a+1)b} \\
  &=& \frac{2 - (a+1)b}{b(a + 2)(1 - a)} \\
\end{eqnarray*}

Or, to solve for $ x $ (and therefore $ y $), we have:
\begin{eqnarray*}
  a(a + 1)x + abz &=& a \\
  2x + abz &=& b \\
  (a(a+1) - 2)x &=& a - b \\
  x &=& \frac{a - b}{a(a+1) - 2} \\
  x &=& \frac{a - b}{(a-1)(a+2)}
\end{eqnarray*}

So the answer is $ \left(\frac{a - b}{(a-1)(a+2)}, \frac{a - b}{(a-1)(a+2)}, \frac{2 - (a+1)b}{b(a + 2)(1 - a)} \right)$

\subsection*{Part b}
\subsubsection*{Part i}
When $ a = -2 $, the equations simplifies to
\begin{eqnarray*}
  -2x + y + bz &=& 1 \\
   x - 2y + bz &=& 1 \\
   x + y  -2bz &=& b \\
\end{eqnarray*}

Using the first two equations, we can solve 
\begin{eqnarray*}
  -2x + y + bz &=& 1 \\
   x - 2y + bz &=& 1 \\
  (-2x + y) - (x - 2y) &=& 0 \\
  -3x + 3y &=& 0 \\
  x &=& y
\end{eqnarray*}
Putting it back to equation 1, we have 
\begin{eqnarray*}
  -2x + x + bz &=& 1 \\
            bz &=& 1 + x
\end{eqnarray*}
Putting it in equation 3, we have
\begin{eqnarray*}
  x + x -2 (1 + x) &=& b \\
  b &=& -2
\end{eqnarray*}
So if $ a = -2 $, $ b $ must be $ -2 $. 

Since $ bz = 1 + x $, therefore the solutions are $ (x, x, \frac{1+x}{-2}) $
\subsubsection*{Part ii}
When $ a = 1 $, the equations simplifies to
\begin{eqnarray*}
  x + y + bz &=& 1 \\
  x + y + bz &=& 1 \\
  x + y + bz &=& b \\
\end{eqnarray*}
It is obviously that it is required for $ b = 1 $ in order to be consistent. It is also obvious that the solution is  $ (x, y, (1- x - y)) $

\subsection*{Part c}
When $ b = 0 $, the equations simplifies to
\begin{eqnarray*}
  ax + y &=& 1 \\
   x + ay &=& 1 \\
   x + y &=& 0
\end{eqnarray*}
Using the first two equations, we can solve 
\begin{eqnarray*}
  ax + y &=& 1 \\
  x + ay &=& 1 \\
  (a - 1)x &=& (a - 1)y \\
\end{eqnarray*}
Suppose the equations are consistent so that a solution exists. We consider the two cases $ a \ne 1 $ or $ a = 1 $.
Suppose $ a \ne 1 $, $ x = y $, the last equation implies $ x = y = 0 $, that contradicts the $ ax + y = 1 $
Suppose $ a = 1 $, the first equation reads $ x + y = 1 $ and the last equation reads $ x + y = 0 $, that is also a contradiction. 

In both case, we reach a contradiction, therefore the equations are inconsistent if $ b = 0 $.