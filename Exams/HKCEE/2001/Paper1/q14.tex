\section*{Problem 14}
\subsection*{Part a}
\subsubsection*{Part i}
There is no magic in this part, just calculate, all numbers in the table in round to 3 significant digits.
\begin{center}
\begin{tabular}{ |c|c| }
    \hline
    x    & f(x)    \\
    \hline
    1    & 0       \\
    1.05 & -0.0237 \\
    1.1  & 0.0105  \\
    1.15 & 0.111   \\
    \hline
\end{tabular}
\end{center}

\subsubsection*{Part ii}
Here is the bisection process, to avoid losing precision, all immediate numbers are round to 6 decimal places:

\begin{eqnarray*}
    f(1.05)   &=& -0.0237  \\
     f(1.1)   &=&  0.0105  \\
     f(1.075) &=& -0.01437 \\
    f(1.0875) &=& -0.00394 \\
   f(1.09375) &=&  0.00277 \\
   f(1.09063) &=& -0.00071 \\
   f(1.09219) &=&  0.00100 \\
   f(1.09141) &=&  0.00014 \\
   f(1.09102) &=& -0.00029
\end{eqnarray*}

Therefore the answer is $ 1.091 \cdots $,

\subsection*{Part b}
\begin{enumerate}
  \item{The money deposited on 1997 is compounded 4 times, so we have got $ 1000 * (1+r)^4 $}
  \item{The money deposited on 1998 is compounded 3 times, so we have got $ 1000 * (1+r)^3 $}
  \item{The money deposited on 1999 is compounded 2 times, so we have got $ 1000 * (1+r)^2 $}
  \item{The money deposited on 2000 is compounded 1 times, so we have got $ 1000 * (1+r)^1 $}
\end{enumerate}
The total money is therefore $ 1000 ((1 + r) + (1 + r)^2 + (1 + r)^3 + (1 + r)^4) $, but it is also 5000, so we can solve for $ r $. To start with, we simplify the geometric progression as follows. Let $ g = (1+r) $ and $ S = (1 + r) + (1 + r)^2 + (1 + r)^3 + (1 + r)^4 $, we have:

\begin{eqnarray*}
         S &=& g + g^2 + g^3 + g^4   \\
        gS &=& g^2 + g^3 + g^4 + g^5 \\
    S - gS &=& g - g^5               \\
  S(1 - g) &=& g - g^5               \\
         S &=& \frac{g - g^5}{1 - g}
\end{eqnarray*}

Now we can solve:
\begin{eqnarray*}
  1000S = 5000 \\
  S = 5
\end{eqnarray*}

and also
\begin{eqnarray*}
            S &=& \frac{g - g^5}{1 - g} \\
            5 &=& \frac{g - g^5}{1 - g} \\
       5 - 5g &=& g - g^5 \\
  g^5 -6g + 5 &=& 0
\end{eqnarray*}

That is just part a, which we already solved $ g \approx 1.091 $, therefore $ r = g - 1 \approx 0.091 $, or 9.1\%.

