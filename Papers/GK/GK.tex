\documentclass{article}
\usepackage[utf8]{inputenc}
\usepackage{amsmath}
\usepackage[margin=1in]{geometry}
\usepackage{hyperref}

\title{Space-Efficient Online Computation of Quantile Summaries Reading Notes}
\author{Cecilia}
\date{September 2021}

\begin{document}
\maketitle
\section*{About section 2.1}
\subsection*{Before Proposition 1}
Since we are only maintaining a summary. In general we do not know the true rank of an input. Therefore we can only have $ r_{min} $ and $ r_{max} $ as (inclusive) bounds to the real rank.

Unlike array index, rank starts from 1 in this paper. So for $ n $ numbers, the ranks are 1 to $ n $.

The rank bounds for the minimum and maximum is obvious. It is $ r_{min}(v_0) = r_{max}(v_0)= 1 $, $ r_{min}(v_{s-1}) = r_{max}(v_{s-1})= n $

In order to make $ r_{min}(v_i) = \sum\limits_{i \le j}g_i $ work, $ g_0 $ has to be defined to be 1. [Obviously, there is no $ v_{-1} $], so we need an alternative definition. With that, the sum relation follows by telescoping.

\begin{eqnarray*}
  & & \sum\limits_{i \le j}g_i \\
  &=& 1 + \sum\limits_{i=1}^{j}g_i \\
  &=& 1 + \sum\limits_{i=1}^{j}{\left(r_{min}(v_i) -r_{min}(v_{i-1})\right)} \\
  &=& 1 + \sum\limits_{i=1}^{j}{r_{min}(v_i)} - \sum\limits_{i=1}^{j}{r_{min}(v_{i-1})} \\
  &=& 1 + \sum\limits_{i=1}^{j}{r_{min}(v_i)} - \sum\limits_{i=0}^{j-1}{r_{min}(v_{i})} \\
  &=& 1 + \left(\sum\limits_{i=1}^{j-1}{r_{min}(v_i)} + r_{min}(v_j)\right) - \left(r_{min}(v_0) + \sum\limits_{i=1}^{j-1}{r_{min}(v_i)}\right) \\
  &=& 1 + r_{min}(v_j) - r_{min}(v_0) \\
  &=& 1 + r_{min}(v_j) - 1 \\
  &=& r_{min}(v_j)
\end{eqnarray*}

The number of observations that may have fallen between $ v_i $ and $ v_{i-1} $ is $ r_{max}(v_i) - r_{min}(v_{i-1}) - 1 $. The bound is exclusive.
\begin{eqnarray*}
  & & r_{max}(v_i) - r_{min}(v_{i-1}) - 1 \\
  &=& \left(\sum\limits_{j=1}^{i}g_j + \Delta_i\right) - \left(\sum\limits_{j=1}^{i-1}g_j\right) - 1\\
  &=& g_j + \Delta_i - 1
\end{eqnarray*}

$ \sum{g_i} = n $ follows as that is $ r_{min}(v_{s-1}) $

\subsection*{Proposition 1}
In the $ r > n - e $ case, $ r - e \le r_{min}(v_{s-1}) $ is trivial as any rank must be less than $ n $.
$ r + e \ge r_{max}(v_{s-1}) $ is basically $ r \ge n - e $ which is given in the case.

The "It follows that part" in the a bit confusing in writing, it is actually the conclusion of the proof that is coming up right after. Instead, let's assume (for the purpose of contradiction) that $ r - e > r_{min}(v_{j-1}) $ 

In the other case, we have:
\begin{eqnarray*}
  r_{max}(v_j) &>& r + e \\
  r_{min}(v_j) + \Delta_{j} &>& r + e \\
  r_{min}(v_{j-1}) + g_j + \Delta_{j} &>& r + e \\
  r_{min}(v_{j-1}) + g_j + \Delta_{j} &>& r - e + 2e \\
                                      &>& r_{min}(v_{j-1}) + 2e \\
                     g_j + \Delta_{j} &>& 2e 
\end{eqnarray*}
But recall $ e = \max(\frac{g_j + \Delta_{j}}{2}) $, so we have a contradiction and therefore $ r - e \le r_{min}(v_{j-1}) $

\subsection*{After Proposition 1}
The concept of capacity is unclear, what does it mean by $ g_i $ counting certain observations? Looking backward from the algorithm analysis, since $ \Delta_i $ never changes and $ g_i + \Delta_i < 2\epsilon n $, it make sense to think of a tuple has space to grow if $ g_i $ is small.

\subsection*{Band}
The subtraction by $ p \pmod 2^\alpha $ make sure the band boundaries are at powers of two. 

The larger the $ \alpha $, the wider is the band, the smaller are the $ \Delta $ values, the earlier its insertion time.

Notice later in the insertion algorithm, the $ \Delta $ values are set to be the $ \lfloor 2\epsilon n \rfloor $, so the first $ \lfloor \frac{1}{2\epsilon} \rfloor $ observations has $ \Delta = 0 $ when they were inserted.  

Notice the paper said the first $ \frac{1}{2\epsilon} $ observations. That does not make sense as $ \frac{1}{2\epsilon} $ is not an integer.

The exact property is that if two elements were in the same band, they cannot become in different bands as $ n $ increases. A $ \Delta $ value does increase band when $ n $ increases.

\subsection*{Tree representation}
It is unclear what does it mean by children order - the author only specify which node should be the parent of which. Anyway, here is how I interpreted it.

Imagine an algorithm as follow. Scanning from the left to right, for each node, we find its parent and add it to the list of children. Now we have a definition of children ordering.

Suppose in this children list, B goes after A but B's band is larger than A, then B should be the parent of A and therefore they should not be siblings. That proves proposition 3.

For proposition 4, suppose we have a gap in the descendent of $ P $ such that $ R $ has a node $ Q $ before it which is a descendent of $ P $ yet $ R $ is not. Then 

$ R $ must have a rank higher than $ P $, or else it must be a descendent of $ P $.
$ Q $ must have a rank lower than $ P $, or else it cannot be a descendent of $ P $.

Now we $ R $ should be the parent of $ Q $ instead, then $ R $ is not a descendent of $ P $, this contradiction shows the descendent must by contiguous with no gap.

\subsection*{Operations}
It is unclear how compress should be implemented. In particular, how to I walk the descendants?

What does it meant by $ n \equiv 0 \pmod{\frac{1}{2\epsilon}} $ as $ \frac{1}{2\epsilon} $ may not be an integer? Note that since $ \Delta = \lfloor 2\epsilon n \rfloor $, so $ \Delta $ increase by 1 roughly when $ n $ increase by $ \frac{1}{2\epsilon} $, is that what it meant? Do a compress every time when $ \Delta $ increase? Is it possible that COMPRESS is not going to be productive if $ \Delta $ does not increase? Why?

\subsection*{Analysis}
It is unclear why these operations maintain the $\epsilon$-approximate summary. To prove that these operations indeed maintain an $\epsilon$-approximate summary, it suffice to prove by induction, that is, every operations maintain it. We need two facts:

\begin{enumerate}
  \item $ {r_{min}(v_i) \le rank(v_i) \le r_{max}(v_i)} $ (condition 1), and
  \item $ \max(g_i + \Delta_i) \leq \lfloor 2\epsilon n \rfloor $ (condition 2)
\end{enumerate}

\subsubsection*{Insertion}
Denote by $ i $ the index of the newly inserted item. We know that (condition 1) is unchanged for all tuples with index less than $ i $ and (condition 2) holds for all items (including $ i $). So the only thing to worry about it (condition 1) for tuples with index $ \ge i $.

The notation get a bit tricky because $ rank(v_i) $, $ r_{min} $ and $ r_{max} $ for tuples with index $ \ge i $ changes during the insertion, so we will mark the with before and after. Tuple indexes change as well, so when we refer to a tuple, we always use the post insertion index.

It is clear than 
\begin{align*}
    r_{min}(v_{i-1}) &\le rank(v_{i-1}) & \text{Induction hypothesis} \\
                     &\le rank(v_i) - 1 & \text{Earlier tuple is at most 1 before me} \\
r_{min}(v_{i-1}) + 1 &\le rank(v_i) \\
     r_{min}(v_{i})  &\le rank(v_i) & g_i = 1
\end{align*}
Also 
\begin{align*}
rank(v_i) & <     rank_{after}(v_{i+1})  & \text{This is obvious} \\
          & =     rank_{before}(v_{i+1}) + 1 & \text{Insertion increases rank by 1} \\
          & \leq  r_{max-before}(v_{i+1}) + 1 & \text{Induction hypothesis} \\
          & =     r_{min-before}(v_{i+1}) + \Delta_{i+1} + 1 \\
          & =     g_{i+1} + r_{min}(v_{i-1}) + \Delta_{i+1} + 1 \\
          & \leq     r_{min}(v_{i-1}) + 1 + \lfloor 2\epsilon (n-1 )\rfloor & \text{Induction} \\
          & \leq     r_{min}(v_{i-1}) + 1 + \lfloor 2\epsilon n \rfloor \\
          & =        r_{min}(v_{i-1}) + 1 + \Delta_i \\
          & =     r_{min}(v_i) + \Delta_i & g_i = 1 \\
          & =     r_{max}(v_i)
\end{align*}

For all tuples with index $ > i$, their rank increase by 1, and so are the bounds, so we are good there as well.

\subsubsection*{Deletion}
Deletion do not change $ r_{min} $ or $ r_{max} $, so condition 1 is good. In compress, deletion is guard by condition 2, so condition 2 is good as well.

\subsection*{Lemma 1}
The clause "an observation from a band < $\alpha$" is unclear. An observation don't have a band. I believe the correct interpretation of this clause is "an observation that was covered by a band < $\alpha$ any time before now.

\subsection*{Lemma 2}
The proof said there is no more than $ \frac{1}{2\epsilon} $ observations for each $ \Delta $, but that is not always true.

When $ \epsilon = 0.3 $, $ \lfloor 2(0.3)5 \rfloor = \lfloor 2(0.3)6 \rfloor = 3 $. So there are two tuples with $ \Delta =  3 $, but $ \frac{1}{2\epsilon} = \frac{5}{3} < 2 $. 

Maybe we should say there is no more than $ \lceil \frac{1}{2\epsilon} \rceil $ observations for each $ \Delta $.

\subsection*{Lemma 3}
$ m_{min} $ cannot be exactly $ \frac{\cdots}{2\epsilon} $ as it is a fraction, same for $ m_{max} $

I don't think it is possible to have any node between the rightmost child (my interpretation, the one with largest $ j $) and its parent $ V_i $. If such a $ V_k $ has band $ \ge \alpha $, then $ V_k $ would be the parent. Otherwise $ V_k $ must be a descendent of $ V_i $, contradicting the fact $ V_j $ is the rightmost child.

I guess the author meant the leftmost (i.e. the one with least $ j $) instead. Now it is totally possible that $ V_K $ exists, $ \cdots $


\subsection*{Links}
\begin{enumerate}
    \item {\href{http://www.mathcs.emory.edu/~cheung/Courses/584/Syllabus/08-Quantile/Greenwald2.html}{Notes from emory}}
\end{enumerate}


\end{document}
