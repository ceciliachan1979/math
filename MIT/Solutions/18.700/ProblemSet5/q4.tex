\section*{Question 4}
Since we have $ dim(V) $ distinct eigenvalues, the eigenvectors $ v_1, v_2 \cdots v_n $ form a basis, any vectors and be written as a linear combination of them. Denote the eigenvalues of $ v_i $ corresponding to $ S $ to be $ \lambda_{si} $ and corresponding to $ T $ to be $ \lambda_{ti} $.

Consider both $ ST $ applied to an arbitrary vector $ v $ as follow:

\begin{eqnarray*}
  & & ST v \\
  &=& ST (\sum c_i v_i) \\
  &=& \sum c_i ST v_i \\
  &=& \sum c_i S \lambda_{ti}v_i \\
  &=& \sum c_i \lambda_{ti} S v_i \\
  &=& \sum c_i \lambda_{ti} \lambda_{si} v_i \\
\end{eqnarray*}

Similarly, when applied to $ TS $, we have:

\begin{eqnarray*}
  & & TS v \\
  &=& TS (\sum c_i v_i) \\
  &=& \sum c_i TS v_i \\
  &=& \sum c_i T \lambda_{si}v_i \\
  &=& \sum c_i \lambda_{si} T v_i \\
  &=& \sum c_i \lambda_{si} \lambda_{ti} v_i \\
  &=& \sum c_i \lambda_{ti} \lambda_{si} v_i \\
  & & ST v \\
\end{eqnarray*}

Since the two linear transformation behave exactly the same to all vectors in $ V $, these two linear transformation are identical.