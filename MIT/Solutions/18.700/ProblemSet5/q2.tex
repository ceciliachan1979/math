\section*{Question 2}
\subsection*{Part a}
\begin{eqnarray*}
\left(\begin{array}{cc}
1 & 2 \\
1 & 0
\end{array}\right)
\left(\begin{array}{c}
x \\
y
\end{array}\right) &=&
\lambda
\left(\begin{array}{c}
x \\
y
\end{array}\right) \\
\left(\begin{array}{cc}
1 & 2 \\
1 & 0
\end{array}\right)
\left(\begin{array}{c}
x \\
y
\end{array}\right) -
\lambda
\left(\begin{array}{c}
x \\
y
\end{array}\right) &=& 0 \\
\left[
\left(\begin{array}{cc}
1 & 2 \\
1 & 0
\end{array}\right)
- \lambda I
\right]
\left(\begin{array}{c}
x \\
y
\end{array}\right) &=& 0 \\
\det\left[
\left(\begin{array}{cc}
1 & 2 \\
1 & 0
\end{array}\right)
- \lambda I
\right] &=& 0 \\
(1 - \lambda)(-\lambda) - 2 &=& 0 \\
\lambda^2 - \lambda - 2 &=& 0 \\
(\lambda + 1)(\lambda - 2) &=& 0  
\end{eqnarray*}
When $ \lambda = -1 $, we have
\begin{eqnarray*}
\left(\begin{array}{cc}
1 & 2 \\
1 & 0
\end{array}\right)
\left(\begin{array}{c}
x \\
y
\end{array}\right) &=&
-
\left(\begin{array}{c}
x \\
y
\end{array}\right) \\
\left(\begin{array}{c}
x + 2y \\
x
\end{array}\right) &=&
-
\left(\begin{array}{c}
x \\
y
\end{array}\right) \\
\end{eqnarray*}
So we find $ (1, -1)^T $ is an eigenvector corresponding to the eigenvalue -1.

When $ \lambda = 2 $, we have
\begin{eqnarray*}
\left(\begin{array}{cc}
1 & 2 \\
1 & 0
\end{array}\right)
\left(\begin{array}{c}
x \\
y
\end{array}\right) &=&
2
\left(\begin{array}{c}
x \\
y
\end{array}\right) \\
\left(\begin{array}{c}
x + 2y \\
x
\end{array}\right) &=&
\left(\begin{array}{c}
2x \\
2y
\end{array}\right) \\
\end{eqnarray*}
So we find $ (2, 1)^T $ is an eigenvector corresponding to the eigenvalue 2.
\subsection*{Part b}
By inspection:
\begin{eqnarray*}
\left(\begin{array}{c}
3 \\
0
\end{array}\right) &=& 
\left(\begin{array}{c}
2 \\
1
\end{array}\right)
+
\left(\begin{array}{c}
1 \\
-1
\end{array}\right) \\
\left(\begin{array}{c}
1 \\
0
\end{array}\right) &=& 
\frac{1}{3}\left(\begin{array}{c}
2 \\
1
\end{array}\right)
+
\frac{1}{3}\left(\begin{array}{c}
1 \\
-1
\end{array}\right)
\end{eqnarray*}
\subsection*{Part c}
Like part b, we can also have:
\begin{eqnarray*}
\left(\begin{array}{c}
0 \\
3
\end{array}\right) &=& 
\left(\begin{array}{c}
2 \\
1
\end{array}\right)
-2
\left(\begin{array}{c}
1 \\
-1
\end{array}\right) \\
\left(\begin{array}{c}
0 \\
1
\end{array}\right) &=& 
\frac{1}{3}\left(\begin{array}{c}
2 \\
1
\end{array}\right)
-
\frac{2}{3}\left(\begin{array}{c}
1 \\
-1
\end{array}\right)
\end{eqnarray*}
Now we can solve
\begin{eqnarray*}
  & & A^n\left(\begin{array}{c}
0 \\
1
\end{array}\right) \\
  &=& A^n \left[\frac{1}{3}\left(\begin{array}{c}
2 \\
1
\end{array}\right)
-
\frac{2}{3}\left(\begin{array}{c}
1 \\
-1
\end{array}\right)\right] \\
  &=& \frac{1}{3}A^n\left(\begin{array}{c}
2 \\
1
\end{array}\right)
-
\frac{2}{3}A^n\left(\begin{array}{c}
1 \\
-1
\end{array}\right) \\
  &=& \frac{1}{3}2^n\left(\begin{array}{c}
2 \\
1
\end{array}\right)
-
\frac{2}{3}(-1)^n\left(\begin{array}{c}
1 \\
-1
\end{array}\right) \\
  &=& \frac{1}{3}\left(\begin{array}{c}
2^{n+1} + 2(-1)^{n+1} \\
2^n + 2(-1)^n
\end{array}\right) \\
\end{eqnarray*}