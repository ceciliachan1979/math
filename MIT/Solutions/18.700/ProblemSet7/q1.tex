\section*{Question 1}
\subsection*{Part a}
Since $ \{ f_1, \cdots f_n \}$ is a basis, any vector $ v $ can be written as a linear combination of them. Taking the inner product and using the orthogonal property, we have:

\begin{eqnarray*}
  v &=& \sum\limits_{i=1}^n c_i f_i \\
  \langle v, f_k \rangle &=& \sum\limits_{i=1}^n c_i \langle f_i f_k \rangle \\
  \langle v, f_k \rangle &=& c_k \langle f_k f_k \rangle \\  
  c_k &=& \frac{\langle v, f_k \rangle}{\langle f_k f_k \rangle } \\
    v &=& \sum\limits_{i=1}^n \frac{\langle v, f_i \rangle}{\langle f_i f_i \rangle } f_i \\
\end{eqnarray*}

\subsection*{Part b}
The basic idea is the the $ j $ column of $ A $ is what $ T(f_j) $ is, so $ A_{ij} $ is really just the coefficient of $ T(f_j) $ corresponding to the $ f_i $, which can be extracted using inner product.

Consider the matrix form of $ T(\sum\limits_{i=1}^n c_i f_i) $. 
\begin{eqnarray*}
  & & T(\sum\limits_{i=1}^n c_i f_i) \\
  &=& \sum\limits_i (\sum\limits_j A_{ij} c_j) f_i
\end{eqnarray*}
Suppose the vector under transformation is $ f_q $, then $ c_i = \delta_{iq} $, we have:

\begin{eqnarray*}
                       T(f_q) &=& \sum\limits_i A_{iq} f_i \\
  \langle T(f_q), f_p \rangle &=& \langle \sum\limits_i A_{iq} f_i, f_p \rangle \\
                              &=& A_{pq}\langle f_p, f_p \rangle \\
                       A_{pq} &=& \frac{\langle T(f_q), f_p \rangle}{\langle f_p, f_p \rangle}
\end{eqnarray*}

\subsection*{Part c}
Similarly, $ B_{ij} $ is the coefficient of $ T^*(f_j)$ corresponding to $ f_i $, so we have a similar calculation here:

\begin{eqnarray*}
                       T^*(f_q) &=& \sum\limits_i B_{iq} f_i \\
  \langle T^*(f_q), f_p \rangle &=& \langle \sum\limits_i B_{iq} f_i, f_p \rangle \\
                                &=& B_{pq}\langle f_p, f_p \rangle \\
                         B_{pq} &=& \frac{\langle T^*(f_q), f_p \rangle}{\langle f_p, f_p \rangle}
\end{eqnarray*}

\subsection*{Part d}
Let's do more simplification of $ B_{pq} $
\begin{eqnarray*}
   B_{pq} &=& \frac{\langle T^*(f_q), f_p \rangle}{\langle f_p, f_p \rangle} \\
          &=& \frac{\langle f_q, T(f_p) \rangle}{\langle f_p, f_p \rangle} \\
          &=& \frac{\overline{\langle T(f_p), f_q \rangle}}{\langle f_p, f_p \rangle}
\end{eqnarray*}
So 
\begin{eqnarray*}
   B_{pq}\langle f_p, f_p \rangle &=& \overline{\langle T(f_p), f_q \rangle} \\
                                  &=& \overline {A_{qp}\langle f_q, f_q \rangle} \\
                           B_{pq} &=& \overline {A_{qp}}\frac{\langle f_q, f_q \rangle}{\langle f_p, f_p \rangle}
\end{eqnarray*}

This would be a simple conjugate transpose if the basis is orthonormal instead of just orthogonal.