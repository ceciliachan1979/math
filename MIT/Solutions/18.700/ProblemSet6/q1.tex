\section*{Question 1}
\subsection*{Part a}
It is unclear what "preserve" means in this context, but I can prove that the range is a subset of $ P_m(R) $. 

\begin{eqnarray*}
  & & deg(D p) \\
  &=& deg(p_2 \frac{d}{dx^2}p + p_1 \frac{d}{dx}p + p_0 p ) \\
  &\le& max(deg(p_2 \frac{d^2p}{dx^2}, deg(p_1 \frac{dp}{dx}), deg(p_0 p )) \\
  &\le& max(deg(p_2)+deg(\frac{d^2p}{dx^2}), deg(p_1)+deg(\frac{dp}{dx}), deg(p_0)deg(p)) \\
  &\le& max(2+(m-2), 1+(m-1), 0 + m) \\
  &=& m  
\end{eqnarray*}

Therefore the range of $ D $ is a subset of $ P_m(R) $.

\subsection*{Part b}
Consider $ D(x^k) $, we have these special cases:

\begin{eqnarray*}
  D x^0 &=& f \\
  D x^1 &=& (d + f)x + e
\end{eqnarray*}

For $ k \ge 2 $, we have

\begin{eqnarray*}
  & & D(x^k) \\
  &=& (ax^2 + bx + c)\frac{d^2}{dx}x^k + (dx + e)\frac{d}{dx}x^k + fx^k \\
  &=& (ax^2 + bx + c)k(k-1)x^{k-2} + (dx + e)kx^{k-1} + fx^k \\
  &=& k(k-1)(ax^k + bx^{k-1} + cx^{k-2}) + k(dx^k + ex^{k-1}) + fx^k \\
\end{eqnarray*}

The diagonal entries of the matrix correspond to the coefficient of $ x_k $. The first two are $ f $ and $ d + f $, the remaining ones are $ k(k-1)a + kd +f $ for $ k \ge 2 $.
\subsection*{Part c}
Suppose $ 0 < a < d $, we see those diagonal entries are distinct, in fact, it is an increasing sequence. In this case, the operator is diagonalizable (because the diagonal entries are eigenvalues, distinct eigenvalues implies we have $ m $ linearly independent eigenvectors and they will span $ P_m(R) $.

\subsection*{Part d}
In case $ a = 1. b = -1, c = 0, d = 2, e = -1, f = 0 $, we have the polynomials:

\begin{eqnarray*}
  D(x^0) &=& 0 \\
  D(x^1) &=& 2x - 1 \\
  D(x^2) &=& 2(2-1)(x^2 - x^1) + 2(2x^2 - x^1) \\
         &=& 6x^2 - 4x \\
  D(x^3) &=& 3(3-1)(x^3 - x^2) + 3(2x^3 - x^2) \\
        &=& 12x^3 - 9x^2 \\
\end{eqnarray*}

The correspond to the matrix:
\begin{eqnarray*}
  & & M(D) \\
  &=& \left(\begin{array}{cccc}
    0 & -1 &  0 & 0  \\
    0 & 2  & -4 & 0  \\
    0 & 0  &  6 & -9 \\
    0 & 0  &  0 & 12
      \end{array}\right)
\end{eqnarray*}

The eigenvalues are obviously the ones on the diagonal, finding the eigenvectors is just a tedious exercise that is best done by the computer, here are the eigenvectors:

\begin{eqnarray*}
  D(1) &=& 0 \times 1 \\
  D(2x - 1) &=& 2 \times (2x - 1) \\
  D(6x^2 - 6x + 1) &=& 6 \times (6x^2 - 6x + 1) \\
  D(20x^3 - 30x^2 + 12x - 1) &=& 12 \times (20x^3 - 30x^2 + 12x - 1)
\end{eqnarray*}