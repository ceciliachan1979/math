\section*{Question 2}
The key idea is that the $ F_9 $ elements are $ 0, 1, 2, x, x + 1, x + 2, 2x, 2x + 1 $ and $ 2x + 2$. Once we have that, the rest is simply doing all the 9 x 9 multiplication. Here are the results:
\begin{eqnarray*}
  & & \left(
        \begin{array}{ccccccccc}
0 & 0 & 0 & 0 & 0 & 0 & 0 & 0 & 0 \\
0 & 1 & 2 & x & x + 1 & x + 2 & 2x & 2x + 1 & 2x + 2 \\
0 & 2 & 1 & 2x & 2x + 2 & 2x + 1 & x & x + 2 & x + 1 \\
0 & x & 2x & 2 & x + 2 & 2x + 2 & 1 & x + 1 & 2x + 1 \\
0 & x + 1 & 2x + 2 & x + 2 & 2x & 1 & 2x + 1 & 2 & x \\
0 & x + 2 & 2x + 1 & 2x + 2 & 1 & x & x + 1 & 2x & 2 \\
0 & 2x & x & 1 & 2x + 1 & x + 1 & 2 & 2x + 2 & x + 2 \\
0 & 2x + 1 & x + 2 & x + 1 & 2 & 2x & 2x + 2 & x & 1 \\
0 & 2x + 2 & x + 1 & 2x + 1 & x & 2 & x + 2 & 1 & 2x \\
        \end{array}
      \right) \\
\end{eqnarray*}

Of course, I am not going to do that 81 multiplications myself, the work is done by these simple python code:

\begin{verbatim}
e = []

def mul(x,y):
    (a,b) = x
    (c,d) = y
    # (ax + b)(cx + d)
    # = acx^2 + (ad + bc)x + bd
    # = (ad + bc)x + (bd - ac)
    p = (a * d + b * c) % 3
    q = (b * d - a * c + 3) % 3
    if (p == 0):
        return q
    elif p == 1:
        if q == 0:
            return "x"
        else:
            return "x + %s" % q
    else:
        if q == 0:
            return "2x"
        else:
            return "2x + %s" % q

for c in range(0, 3):
    for d in range(0, 3):
        e.append((c,d))

for p in e:
    for q in e:
        print(mul(p,q),end=" ")
    print()

\end{verbatim}