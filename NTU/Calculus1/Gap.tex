\section*{Lemma 1: $(a + b) \ge 2\sqrt{ab}$ for $a, b \in R \ge 0$}

$ (\sqrt{a} - \sqrt{b})^2 = a - 2\sqrt{ab} + b \ge 0 $

Equality holds if and only if $ a = b $

\section*{Lemma 2: if $ a \in \mathbb{Q} > \sqrt{2} $, then there exist $ b \in \mathbb{Q} $ such that $ a > b > \sqrt{2} $.}

Consider $ a + \frac{2}{a} $, since $ a \ne \frac{2}{a} $, by lemma 1, we have $ a + \frac{2}{a} > 2\sqrt{a\frac{2}{a}} = 2 \sqrt{2}$

Therefore, let $ b = \frac{1}{2}\left(a + \frac{2}{a}\right) $, we have $ b > \sqrt{2} $.

Also, since $ a > \sqrt{2} $, $ \frac{2}{a} < \sqrt{2} < a$. Therefore, $ b $ , as the average of them is less than $ a $.

\section*{Lemma 3: if $ a \in \mathbb{Q} < \sqrt{2} $, then there exist $ b \in \mathbb{Q} $ such that $ a < b < \sqrt{2} $.}

In this case we have $ \frac{2}{a} > \sqrt{2} $, by lemma 2, there exist $ c $ such that $ \frac{2}{a} > c > \sqrt{2} $, which implies $ a < \frac{2}{c} < \sqrt{2} $.

\section*{Dedekind cuts on $\mathbb{Q} $ does not have the gapless property}

Consider the cut $ A = \{ a \in \mathbb{Q} | a^2 < 2\} $ and $ B = \mathbb{Q} - A $. 

Suppose $ A $ has a maximum $ M $, lemma 3 implies there exists $ M' \in \mathbb{Q} $ such that $ M < M' < \sqrt{2} $ so that $ M' \in A $ but is bigger than the maximum. Therefore $ A $ don't have a maximum.

Similarly, $ B $ does not have a minimum by lemma 2.

This violates the gapless property because none of the required conditions are true.

\section*{Lemma 4: Lower bounded integers does have a minimum}

Proof: Suppose $ A $ is a set of integers with a lower bound $ L $, consider the set of integers $ S = \{b \in \mathbb{Z} | b - L \in A\}$, so $ S $ is a set of natural numbers. By the well ordering property, $ S $ has a minimum $ M' $, so $ M = M' - L $ is a minimum for $ A $.


\section*{Lemma 5: Upper bounded integers does have a maximum}

Proof: Consider the set $ R = \{b \in \mathbb{Z} | -b \in A\} $, $ R $ is now lower bounded and have a minimum $ M' $, and therefore $ -M' $ is the maximum of $ A $.


\section*{Dedekind cuts on $\mathbb{Z} $ does not have the gapless property}

Any cut on the integers will result in a pair of set of integers with an upper bound and a lower bound, these set will have both a minimum and a maximum, so it violates the gapless property because both properties are true (instead of exactly 1)

\section*{Dedekind cuts on $\mathbb{R} $ does have the gapless property}

Suppose $ A, \mathbb{R} - A $ is a Dedekind cut, $ A $ has a upper bound and $ B $ has a lower bound. By the completeness of $ \mathbb{R} $. $ \sup{A} $ and $ \inf{B} $ exists.

They should be equal. Otherwise, consider the average $ t = \frac{\sup{A} + \inf{B}}{2} $ of them.

In case $ \sup{A} < \inf{B} $, $ t \notin A $ because $ t > \sup A  $ and also $ t \notin B $ because $ t < \inf B $, that is a contradiction.

In the other case $ \sup A > \inf{B} $, $ t \in A $ because $ t < \sup A $ but also $ t \in B $ because $ t > \inf B $. But then $ t $ should not be in $ B $ because it isn't greater than all elements in $ A $.

Since $ A \cup B = \mathbb{R} $, either $ t \in A $ or $ t \in B $. In case $ t \in A $, $ A $ has a maximum $ t $, but $ B $ will not have a minimum. Suppose $ B $ does have a minimum $ m $, $ \frac{m + t}{2} $ will be in $ B $ but smaller than the minimum, contradiction. Similarly, in case $ t \in B $, $ B $ will have a minimum but $ A $ will not have a maximum. 

That shows exactly one of the required condition is true, and therefore satisfy the gapless property. 