\documentclass{article}
\usepackage[utf8]{inputenc}
\usepackage{amsmath}

\title{Bernoulli}
\author{Cecilia}
\date{June 2020}

\begin{document}
\maketitle
\section{Power sum}
Trying to derive the coefficients of $ \sum\limits_{a=1}^{n-1} a^k $ as a polynomial of $ n $ as a recurrence relation. [We use $ n - 1 $ instead of $ n $ to make the telescoping easier]
\begin{eqnarray*}
& & n^{k+1} - 1 \\
  &=& \sum\limits_{a=1}^{n-1} \left((a + 1)^{k+1} - a^{k+1}\right) \\
  &=& \sum\limits_{a=1}^{n-1} \left(\sum\limits_{j=0}^{k+1}C_j^{k+1}a^j - a^{k+1}\right) \\
  &=& \sum\limits_{a=1}^{n-1} \sum\limits_{j=0}^{k}C_j^{k+1}a^j \\
  &=& \sum\limits_{a=1}^{n-1} \left(\sum\limits_{j=0}^{k-1}C_j^{k+1}a^j + (k+1)a^k \right) \\
  &=& \sum\limits_{j=0}^{k-1}C_j^{k+1}\sum\limits_{a=1}^{n-1} a^j + (k+1)\sum\limits_{a=1}^{n-1} a^k \\  
  (k+1)\sum\limits_{a=1}^{n-1} a^k &=& n^{k+1} - 1 - \sum\limits_{j=0}^{k-1}C_j^{k+1}\sum\limits_{a=1}^{n-1} a^j \\
\end{eqnarray*}
At this point, we are already representing $ (k+1)\sum\limits_{a=1}^{n-1} a^k $ as a combination of $ \sum\limits_{a=1}^{n-1} a^j $, so we can start letting $ \sum\limits_{a=1}^{n-1} a^p = \sum\limits_{m=0}^{p+1}D_{p+1-m}^kn^m $
\begin{eqnarray*}
  (k+1)\sum\limits_{m=0}^{k+1}D_{k+1-m}^kn^m &=& n^{k+1} - 1 - \sum\limits_{j=0}^{k-1}C_j^{k+1}\sum\limits_{m=0}^{j+1}D_{j+1-m}^jn^m \\
  &=& n^{k+1} - 1 - \sum\limits_{m=1}^{k}\sum\limits_{j=m-1}^{k-1}C_j^{k+1}D_{j+1-m}^jn^m - \sum\limits_{j=0}^{k-1}C_j^{k+1}D_{j+1}^jn^0 \\
  (k+1)D_0^k &=& 1 \\
  (k+1)D_m^k &=& -\sum\limits_{j=k-m}^{k-1}C_j^{k+1}D_{m+j-k}^j  \rm {, for } 1 \le m \le k \\
  (k+1)D_{k+1}^k &=& -1 -\sum\limits_{j=0}^{k-1}C_j^{k+1}D_{j+1}^j
\end{eqnarray*}

\section{To Bernoulli's number (2023)}
We have a fixed 1 for the most significant term, The recurrence equation for the constant term can be easily argued that it is 0 for all but the first equation.

For the middle terms, I wanted to prove that $ \frac{D^k_m(k+1)}{C^{k+1}_m} $ are the Bernoulli numbers and are independent of $ k $. We start with the middle equation. Note that the sum limits depend on $ k $, so let's start with eliminating that dependency first by letting $ r = j - k $

\begin{eqnarray*}
  (k+1)D_m^k &=& -\sum\limits_{j=k-m}^{k-1}C_j^{k+1}D_{m+j-k}^j \\
             &=& -\sum\limits_{r=-m}^{-1}C_{k+r}^{k+1}D_{m+(k+r)-k}^{k+r} \\
             &=& -\sum\limits_{r=-m}^{-1}C_{k+r}^{k+1}D_{m+r}^{k+r} \\
\end{eqnarray*}

Now we make Bernoulli's numbers on both sides

\begin{eqnarray*}
  \frac{D^k_m(k+1)}{C^{k+1}_m} &=& -\sum\limits_{r=-m}^{-1}\frac{C_{k+r}^{k+1}}{C^{k+1}_m}\frac{C^{k+r+1}_{m+r}}{(k+r+1)}\frac{D_{m+r}^{k+r}(k+r+1)}{C^{k+r+1}_{m+r}} \\
\end{eqnarray*}

Turn out the coefficient can be greatly simplified.

\begin{eqnarray*}
  & & \frac{C_{k+r}^{k+1}}{C^{k+1}_m}\frac{C^{k+r+1}_{m+r}}{(k+r+1)} \\
  &=& \frac{(k+1)!}{(k+r)!(-r+1)!} \times \frac{(k+r+1)!}{(m+r)!(k-m+1)!} \times \frac{m! (k - m + 1)!}{(k+1)!} \times \frac{1}{(k+r+1)} \\
  &=& \frac{m!}{(-r+1)!(m+r)!}
\end{eqnarray*}

The most important piece is that the coefficient is independent of $ k $. That means for each $ k $, we are solving essentially the same recurrence, so we will get the same answer.

With that, we can change our notation and denote $ B_m = \frac{D^k_m(k+1)}{C^{k+1}_m} $ and the equation works for all $ k $. The recurrence becomes:

\begin{eqnarray*}
  B_m &=& -\sum\limits_{r=-m}^{-1} \frac{m!}{(-r+1)!(m+r)!} B_{m+r} \\
    0 &=& -\sum\limits_{r=-m}^{0} \frac{m!}{(-r+1)!(m+r)!} B_{m+r} \\
      &=& \sum\limits_{r=-m}^{0} \frac{m!}{(-r+1)!(m+r)!} B_{m+r} \\
      &=& \sum\limits_{i=0}^{m} \frac{m!}{(i + 1)!(m-i)!} B_{m-i} \\
      &=& \sum\limits_{i=0}^{m} \frac{C_i^m}{i+1} B_{m-i} \\
\end{eqnarray*}

\end{document}
