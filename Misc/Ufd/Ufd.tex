\documentclass{article}
\usepackage[utf8]{inputenc}
\usepackage{amsmath}
\usepackage[margin=1in]{geometry}
\usepackage{hyperref}

\title{Unique Factorization Domain}
\author{Cecilia}
\date{October 2021}

\begin{document}
\maketitle
\section*{Overview}
The goal of this document is to prove that polynomials of several variables over a unique factorization domain is a unique factorization domain.

\section{Primes and irreducibles}
In kids math, it is often said that a number if prime if it is not 0, not 1 and have no non-trivial factors beside 1 and itself. This is misleading because this is the definition of being irreducible, not being prime.

An element of an integral domain is irreducible if it is not zero, not a unit, and any factorization of it involves a unit.

An element $ p $ of an integral domain is prime if $ p $ divides $ ab $, then $ p $ divides $ a $ or $ p $ divides $ b $.

A prime is necessarily irreducible. If not, $ p = ab $ where both $ a $ and $ b $ are not units. Now if $ p $ divides $ a $, then $ cp = a \implies bcp = ab = p $, meaning we have a multiplicative inverse $ c $ for $ b $. This is a contradiction because we said $ b $ is not a unit. That shows $ p $ does not divide $ a $, similarly, $ p $ does not divide $ b $ as well. But $ p = ab $ divides $ ab $, so the fact that $ p $ does not divide $ a $ and $ p $ does not divide $ b $ contradicts the fact that $ p $ is prime. So a prime element is necessarily irreducible.

Does an irreducible element necessarily prime? Only in a unique factorization domain.

\section{Euclidean $ \implies $ Principal Ideal Domain}
Polynomials over a field $ F[x] $ forms and Euclidean domain. As a first step, we would like to show that an Euclidean domain is always a principal ideal domain. Given an ideal, we can find a single generator as follow:

\begin{enumerate}
    \item Pick an element, call it $ g $.
    \item If $ g $ generates the ideal, stop
    \item Otherwise, pick an ideal element $ f $ that is not generated by $ g $, replace $ g $ by the $ \gcd(f, g) $. 
\end{enumerate}

Suppose the algorithm stops, the ideal is generated by $ g $. The algorithm must stop because every time we compute the gcd, the Euclidean norm reduces and it is lower bounded by 0, so eventually it has to stop.


\section{Principal Ideal Domain $ \implies $ Noetherian}
Next, we would like to show a principal ideal domain is necessarily Noetherian. A ring is Noetherian if any chain of ideal inclusion must terminate (i.e. $ I_1 \subseteq I_2 \subseteq I_3 \cdots $ must end with $ I_n = I_{n+1} = \cdots $ for some $ n $. For a principal ideal domain, that must be true because $ I_1 \cup I_2 \cup \cdots $ is an ideal and therefore must be a principal ideal $ (x) $ and therefore there exists $ n \in  \mathbf{Z} $ such that $ x \in I_n $ and $ I_{>n} $ must be all the same $ (x) $

\section{Principal Ideal domain $ \implies $ Unique Factorization Domain}
Let $ D $ be a Principal Ideal domain, given an element $ f \in D $. 

\begin{enumerate}
    \item If $ f $ is irreducible, stop
    \item Otherwise factorize it and continue factorizing.
\end{enumerate}
Again, such an algorithm must stop, because if not, this will generate an infinite chain of ideal that does not stop, contradicting the fact that a Principal Ideal domain is Noetherian.

We have just shown that an element in a principal Ideal domain must factor into irreducibles. Now we wanted to show irreducibles are prime.

Let $ p $ be a irreducible that divides $ ab $ but does not divide $ a $ or $ b $.Now consider the ideal generated by $ p $ and $ a $, such an ideal must be generated by a single $ c $.

Now $ c $ cannot be $ p $, otherwise $ p $ divides $ a $. Now since $ p $ is irreducible, $ p = c x $ and $ c \ne p $ implies $ c $ is a unit. We have 

\begin{eqnarray*}
  mp + na &=& c \\
  c^{-1}mpb + c^{-1}nab &=& b \\
\end{eqnarray*}

Now the left hand side is divisible by $ p $, so it is right hand side, and now we have a contradiction.

A factorization into primes is necessarily unique. 

\section*{The constant}
Any element $ r $ in $ R $ is a unit in $ F $, therefore, while $ r $ admits a unique factorization, it doesn't factorize the same way in $ F $. Therefore we wanted to exclude any non unit (in $ R $) constant factor first. To do that, we introduce the primitive part. The primitive part of a polynomial in $ f \in F[x] $ is polynomial $ f' \in R[x] $ where the gcd of the coefficient is 1 where $ f = c f' $ for some $ c \in F $. 

A polynomial in $ F[x] $ always have a unique (up to associates) primitive part. We can multiply the polynomial with the common denominator, and then we can factor out any common factor of the coefficients. This has to be unique up to associate because $ f = c f' $ won't hold otherwise. 

We denote the primitive part of $ f $ by $ pp(f) $

\section*{Properties of the primitive polynomials}
The product of two primitive polynomials is always primitive. We can argue that by considering $ f(x) = f_0 + f_1 x^1 + \cdots f_n x^n $ and similar for $ g $. If $ fg $ is not primitive, then there exists an irreducible $ p $ divides all coefficients of $ fg $. Now consider $ r $ to be the maximum integer such that $ f_r $ is not divisible by $ p $, similarly $ s $ is the maximum integer such that $ g_s $ is not divisible by $ p $. Such a pair must exists because otherwise $ f $ and $ g $ are not primitive. But then the coefficient of $ x^{r+s} = \sum\limits{p + q = r + s}f_p g_q $ is not divisible by $ p $ because $ f_r g_s $ is not divisible by $ p $ but every other terms is. So we reach a contradiction and $ fg $ must also be primitive.

If a polynomial $ f \in R[x] $ is primitive, then it is irreducible if and only if it is also irreducible in $ F[x] $. Suppose $ f = gh $ in $ F[x] $, then $ f = c pp(g)pp(h) $. The primitive parts are in $ R[x] $, so $ c \in R $ as well. In fact, $ c $ must be a unit because $ pp(g)pp(g) $ and $ f $ are both primitive, so $ f $ is not irreducible in $ R[x] $. Similarly if $ f = gh $ in $ R[x] $, then $ f $ is not irreducible in $ F[x] $ as well.

\section*{Unique factorization}
These properties are extremely useful. For any polynomial $ f \in R[x] $, we claim that $ f = r pp(f) $. $ r $ factorizes into $ r_1 r_2 \cdots r_n $ because R is a unique factorization domain. $ pp(f) $ factorizes into $ f_1 f_2 \cdots f_n $ in $ F[x] $, this factorization can be converted into $ c pp(f_1) pp(f_2) \cdots pp(f_n) $, again, we argue $ c $ must be a unit because both sides are primitive. 

Now $ f = r_1 r_2 \cdots r_n c pp(f_1) pp(f_2) \cdots pp(f_n) $. The $ r_* $ are irreducible for sure. we already show $ pp(f_*) $ are also irreducible. 

Since this is a factorization in $ F[x] $, if we had an alternative factorization, then we know that the constant part has to be the same. If they are not, either we contradict the unique factorization property of $ R $ or the polynomial part contains a non primitive polynomial, which can be further factorized. If the polynomial part is not the same, then that would be an alternative factorization in $ F[x] $ because that alternative factorization on the polynomial part must be primitive, so they cannot be associates (even in $ F[x] $).

Now we proved that $ R[x] $ is a unique factorization domain!

\section*{Last, but not least}
Suppose $ F $ is a field, $ F[x] $ is a unique factorization domain, then $ F[x, y] $ can be regarded as $ F[x][y] $, which by our theorem above, is a unique factorization domain, we can add any number of variables as we wish!

\end{document}
