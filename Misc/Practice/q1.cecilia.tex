\section*{Algebra Problem 1}
\subsubsection*{Problem}
The problems can be found \href{https://www.math.hkust.edu.hk/~makyli/190_2010Sp/problemBk.pdf}{here} on page 7.

\subsubsection*{Solution}
This is a rather brute force solution, I guess better solution exists.

\begin{align*}
  p(x) + 1 &| (x - 1)^3                                                                           \\
  p(x) + 1 &= (x - 1)^3(ax^2 + bx + c)                                                            \\
           &= (x^3 - 3x^2 + 3x - 1)(ax^2 + bx + c)                                                \\
           &= ax^5 + (-3a + b)x^4 + (3a - 3b + c)x^3 + (-a + 3b - 3c)x^2 + (-b + 3c)x - c        \\
      p(x) &= ax^5 + (-3a + b)x^4 + (3a - 3b + c)x^3 + (-a + 3b - 3c)x^2 + (-b + 3c)x + (-c - 1)
\end{align*}

\begin{align*}
  p(x) - 1 &| (x + 1)^3                                                                       \\
  p(x) - 1 &= (x + 1)^3(dx^2 + ex + f)                                                        \\
           &= (x^3 + 3x^2 + 3x + 1)(dx^2 + ex + f)                                            \\
           &= dx^5 + (3d + e)x^4 + (3d + 3e + f)x^3 + (d + 3e + 3f)x^2 + (e + 3f)x + f       \\
      p(x) &= dx^5 + (3d + e)x^4 + (3d + 3e + f)x^3 + (d + 3e + 3f)x^2 + (e + 3f)x + (f + 1) \\
\end{align*}

Match coefficients, we have:

\begin{align*}
               a &= d             \\
       (-3a + b) &=  (3d + e)     \\
   (3a - 3b + c) &= (3d + 3e + f) \\
  (-a + 3b - 3c) &= (d + 3e + 3f) \\
      (-bx + 3c) &= (ex + 3f)     \\
        (-c - 1) &= (f + 1)
\end{align*}

Solving, get:

\begin{align*}
  a &= \frac{-3}{8} \\
  b &= \frac{-9}{8} \\
  c &= \frac{-8}{8} \\
  d &= \frac{-3}{8} \\
  e &= \frac{ 9}{8} \\
  f &= \frac{-8}{8}
\end{align*}

Substitute and expand, and now we get

\begin{align*}
  p(x) &= -\frac{3x^5}{8} + \frac{10x^3}{8} - \frac{15x}{8}
\end{align*}

Footnote, the solution has many interesting property such as the symmetry in the $ a, b, c, d, e, f $ and the zeros in the even power terms. Something that suggest there might be some symmetry that I could have used.
