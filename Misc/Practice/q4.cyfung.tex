\section*{Algebra Problem 4, Solution by CY}
\subsubsection*{Problem}
The problems can be found \href{https://www.math.hkust.edu.hk/~makyli/190_2010Sp/problemBk.pdf}{here} on page 7.

\subsubsection*{Solution}

The quadratic functions can be written as $f(x) = \alpha ((x-A)^2 - B)$, $g(x) = \beta ((x-C)^2 - D)$, $h(x) = \gamma ((x-E)^2 - F)$. 

\begin{eqnarray*}
    f(g(h(x))) =& f(\quad \beta \gamma^2(((x-E)^2-F-\frac{C}{\gamma})^2 - \frac{D}{\gamma^2}) \quad ) \\
    =& \alpha \beta^2 \gamma^4 [ [ ((x-E)^2-F-\frac{C}{\gamma})^2 - \frac{D}{\gamma^2} - \frac{A}{\beta\gamma^2} ]^2 - \frac{B}{\beta^2\gamma^4} ]. 
\end{eqnarray*}

Consider the case $\alpha = \beta = \gamma = 1$:
\begin{equation*}
    f(g(h(x))) = [((x-E)^2 - F - C)^2 - D - A]^2 - B.
\end{equation*}

We can see for appropriate transformation, WLOG we can assume a solution of f(g(h(x))) = 0 targetting quadratic polynomials f(x), g(x), h(x) is a set of quadratic polynomials with leading coefficient one.

\hspace{3em}

Solving the equation $ 0 = [((x-E)^2 - F - C)^2 - D - A]^2 - B $, we got eight roots: $E \pm \sqrt{F+C \pm \sqrt{D+A \pm \sqrt{B}}}$.

Assume the parameters are real and no complex roots are involved.

Since square root must be larger than or equal to 0, we can order the eight roots from smallest to largest:

\begin{eqnarray*}
    E - \sqrt{F+C + \sqrt{D+A + \sqrt{B}}} \\ 
    E - \sqrt{F+C + \sqrt{D+A - \sqrt{B}}} \\ 
    E - \sqrt{F+C - \sqrt{D+A - \sqrt{B}}} \\ 
    E - \sqrt{F+C - \sqrt{D+A + \sqrt{B}}} \\ 
    E + \sqrt{F+C - \sqrt{D+A + \sqrt{B}}} \\ 
    E + \sqrt{F+C - \sqrt{D+A - \sqrt{B}}} \\ 
    E + \sqrt{F+C + \sqrt{D+A - \sqrt{B}}} \\ 
    E + \sqrt{F+C + \sqrt{D+A + \sqrt{B}}} \\ 
\end{eqnarray*}

The roots must be corresponding to 1,2,3,4,5,6,7,8. From the sum and difference of root differing by the sign after $E$, we got $E=4.5$, 
$ \sqrt{F+C + \sqrt{D+A + \sqrt{B}}} = 3.5 $ for the first pair, $ \sqrt{F+C - \sqrt{D+A + \sqrt{B}}} = 0.5 $ for the fourth pair. (I) 

We get $ \sqrt{F+C + \sqrt{D+A - \sqrt{B}}} = 2.5 $ for the second pair, $ \sqrt{F+C - \sqrt{D+A - \sqrt{B}}} = 1.5 $ for the third pair. (II)

From (I) we get $F+C = (12.25+0.25)/2=6.25$ . From (II) we get $F+C = (6.25+2.25)/2 = 4.25 $, which is impossible.

Therefore it is impossible to find quadratic functions $f,g,h$ satisfying the requirement. 

\hspace{3em}

Assume the parameters are complex: we have to pair up 1,2,3,4,5,6,7,8 to the roots and it is only possible to get $E=4.5$ for any consistent pairing. Then we can show $F+C$ is real, then similarly for $D+A$ and $B$.
Since $B = [((x-E)^2 - F - C)^2 - D - A]^2$ and $x, E, F+C, D+A$ are real, $B \ge 0$. Similarly we get $D+A - \sqrt{B} \ge 0$, etc.. So it goes back to the assumption that the parameters are real and no complex roots are involved.
\qed

\hspace{5em}

As a bonus, consider the following question from \href{https://math.stackexchange.com/questions/1724410/composition-of-three-quadratic-functions}{Math Stack Exchange}:

\begin{quote}
Find quadratic functions $f,g,h$ such that $f(g(h(x)))$ has $-6, -5, -4, -2, 1, 3, 4, 5$ as its roots.
\end{quote}

Using the above method, we obtain $E=-\frac{1}{2}$, $B=810$, $D+A=106$, $F+C=16\frac{1}{4}$.

We can take on infinitely many solutions. For example, let $D=100$, $F=16$, we have 

\begin{eqnarray*}
    h(x) =& (x-3\frac{1}{2})(x+4\frac{1}{2}) \\
    g(x) =& (x-10\frac{1}{4})(x+9\frac{3}{4}) \\
    f(x) =& (x-96)(x+84).
\end{eqnarray*}

Let $D=196$, $F=2\frac{1}{4}$, we have 
\begin{eqnarray*}
    h(x) =& (x+2)(x-1) \\
    g(x) =& x(x-28) \\
    f(x) =& x(x+180),
\end{eqnarray*}

which is the solution given in Stack Exchange.
