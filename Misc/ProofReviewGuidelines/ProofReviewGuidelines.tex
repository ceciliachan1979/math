\documentclass{article}
\usepackage{hyperref}

\title{Proof review guidelines}
\author{Cecilia Chan}
\date{January 2024}

\begin{document}

\maketitle

\section*{Motivation}
The goal of this document is accumulate knowledge obtained from proof reviews. As we get more feedback, we should generalize the knowledge, make it into the check list below, so that we can check our own proofs. 

\section*{About symbols}
For each used in the proof, we should ask ourselves:
\begin{itemize}
    \item Is it defined?
    \item What are its qualifications?
\end{itemize}

\section*{About definitions}
\begin{itemize}
    \item Is it guaranteed to exist?
    \item Is it already defined? Avoid redefinition.
    \item Is the definition used? Is it used much?
    \item Whenever we say for all, what is the domain?
\end{itemize}

\section*{Are the operations used allowed?}
\begin{itemize}
    \item What domain am I working on, real numbers, or just a general field?
    \item Can the denominator be 0?
    \item Avoid using greatest in set where an ordering is not defined.
    \item Be careful of the order of operands, especially so in non-commutative algebra.
\end{itemize}

\section*{Miscellaneous}
\begin{itemize}
    \item Be careful with typos, do we miss a minus sign somewhere?
    \item In a correct proof, every line is a fact, can I find a counter example to any of the lines?
    \item Is it possible to misinterpret?
    \item Sometimes, we say "it is clear" or "it is obvious", it is really clear and obvious?
\end{itemize}

\end{document}
